\section{Related Work}
\label{sec:related}

Many systems have been proposed to help managing distributed
applications. Plush~\cite{Albrecht2007} is a management system
which support deploying, monitoring and controlling
applications centrally. Plush can bootstrap
(self-deployment) its clients automatically, but the
centralized approach limits its scalability. Plush-M~\cite{Topilski2008} further
improved Plush's scalability on managing applications by
replacing star topology in Plush with RanSub~\cite{Kostic2003}, but the
management of Plush clients is still in centralized
approach. SMON addresses the self-management property of
distributed systems. It can be used to manage other
management systems efficiently.

Researchers have been studying self-* properties of
distributed systems for long time.
Self-stabilization~\cite{Dijkstra1974} is a
concept of fault-tolerance in distributed computing. A
self-stabilization system will converge to a legitimate
state from any state. DHTs~\cite{Stoica2001, Ratnasamy2001, Rowstron2001}
are distributed systems
that have self-organizing properties. They will automatically
organize their overlay topologies conforming to predefined
constraints under changing network conditions.
\cite{Yin2008} proposed self-hosting system that acquires
or releases resources dynamically and automatically deploys
the system to acquired resources (as real or virtual
machines) using underling system management tools. SMON is
complementary to these systems and these techniques can be
combined together to build distributed systems which is more
stable and efficient with little human management efforts.

\comment{
DHT (self-organization),

Dependable Self-Hosting Distributed Systems Using Constraints
self-stabalization

19. E.W. Dijkstra, "Self-Stabilizing Systems in Spite of Distributed Control," Comm.
ACM, vol. 17, no. 11, 1974, pp. 643-644.

20. S. Dolev, Self-Stabilization, MIT Press, 2000.

plush LASCO provides improved scalability and fault
tolerance
}



% vim:foldmethod=marker:textwidth=60
