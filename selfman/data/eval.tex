\section{Evaluation}
\label{sec:eval}

We conducted experiments on PlanetLab platform to evaluate the
performance of BON. Planet-Lab is a global research network
consisting of 800+ nodes at 400+ sites in the world.
Specifically, we provide experimental results showing that BON
is efficient at deploying itself and achieves good scalability,
the overhead caused by race condition is small and the active
state transition is fast.

\subsection{Self-deployment Evaluation}

Self-deployment process is the core of the BON system and we
evaluate it extensively and shows the results in this section.

\begin{figure}
\centering
\includegraphics[scale=0.618]{self-deploy.png}
\caption{Progress of self-deployment process on 159
nodes}
\label{fig:self-deploy}
\end{figure}


First, we evaluate the performance of self-deployment process on
159 Planet-Lab nodes. An instance of BON peer is started and the
list of 159 nodes are given. A BON peer detects other 5 nodes
for every 10 seconds. We finally collected timing data from 154
nodes out of 159 nodes. There are 5 nodes cannot be connected at
data-collecting stage and their data are omitted in the result.
This kind of failure is common for distributed systems
especially when the scale is large. Figure~\ref{fig:self-deploy}
summarizes the progress of self-deployment process.  We can see
that 90\% (143) of nodes are deployed successfully with a BON
peer within 149 seconds. The median deployment time is 93
seconds while the largest one is 533 seconds. 


During the self-deployment process, there are cases that multiple
BON peers try to deploy an instance on the same node and this
can cause extra overhead. The overhead wastes network
bandwidth and the storage at the deployed node. We evaluate the
overhead by count the simultaneous deployment happened at each
node. The result is shown in Figure~\ref{fig:overhead}. We can
see that for 43.71\% of nodes there is exactly one deployment
on each of them. Obviously, most of nodes have a deployment
overhead within 10. These nodes count for 92.04\% of all nodes
and the average overhead of these nodes is 2.31. The average
overhead for all nodes is 4.30 which is an acceptable number. 

\begin{figure}
\centering
\includegraphics[scale=0.5]{overhead.png}
\caption{Overhead introduced by race condition during
self-deployment process}
\label{fig:overhead}
\end{figure}


To evaluate the scalability of self-deployment process, we deploy
a new instance of BON system at a different scale (24 nodes) and
compare the performance of self-deployment process.
Table~\ref{tbl:scalability} summarizes the statistics results.
We can see from the table that the scalability is quite well.
While scale difference between two systems is about 6.6 (159/24), the
90-percentile deployment time is only 1.75 (149/85) times of
difference, while the median value is 1.41 (82/58) times of
difference.

\begin{table}
\centering
\begin{tabular}{|l|c|c|}
\hline
  & 24 nodes & 159 nodes\\
\hline
median & 58 & 82 \\
\hline
90\% percentile & 85 & 149 \\
\hline
Final & 103 & 533 \\
\hline
\end{tabular}
\caption{Comparison of self-deployment process at different scales in seconds}
\label{tbl:scalability}
\end{table}

\subsection{Active State Transition Performance}

\begin{figure}
\centering
\includegraphics[scale=0.618]{state-trans.png}
\caption{Progress of state transition from inactive to active}
\label{fig:state-transition}
\end{figure}

State transition between active state and inactive state
is an important part of BON system and we evaluate its
performance here. We turn the state of BON system deployed on 159
Planet-Lab nodes from inactive to active.
The progress of state transition time is shown in
Figure~\ref{fig:state-transition} and some statistics is
summarized in Table~\ref{tbl:state-transition} for convenience. It can be
concluded that the
state-transition is effective and efficient. For 90\% percentile
of peers, the states are changed after 143 seconds and the
median state transition time is 37 seconds.


\begin{table}
\centering
\begin{tabular}{|l|c|}
\hline
median & 37 \\
\hline
90\% percentile & 143 \\
\hline
\end{tabular}
\caption{Statistics for state transition experiment in seconds}
\label{tbl:state-transition}
\end{table}

