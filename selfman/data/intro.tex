\section{Introduction}

\note{management is a critical problem}

Distrbuted systems are at the heart of today's Internet
services (\note{cloud?}). While the cost of commodity
computers continues to drop, it is common for large scale
clusters to contain thousands of computers, such as
PlanetLab, Amazon EC2 and Teragrid. These platforms can
provide petabytes of data storage and hundreds of teraflops
of computing capabilities that boost performance of
distributed applications significantly. However, the
applications must be carefully designed to scale to
thousands of computers and fully utilize the resources.
%handle frequent and inevitable failures at host,
%application and network levels.

After distributed application is designed and implemented,
an important problem is how to manage it effectively and
efficiently on distributed platforms. Managing a distributed
application involves several tasks. The application must be
first deployed to a set of computers. After the application
is configured and started, its status need to be monitored
for detecting and recovering from failures, ensuring
sustained performance or just for collecting runtime data
and statistics. Operators may also want to send control
commands to control the application on demand, such as fine
tunning application performance.

The distributed application management system is designed to
simply tasks involved in deploying and maintaining
distributed application. To efficiently manage distributed
applications at large scale, the manangement system is
designed in distributed approaches. It consists of many
``peers'' \note{need clarification? client/server} on each
machine where distributed application will be deployed. The
peers work cooperately to disseminate application's
installation package and required data. After the
applicaiton is started, each peer can monitor applicaiton's
processes within the same machine box, and it responds to
messages sent from other peers and user.

The distributed design makes management system a distributed
application in essential and it introduces an important
problem: the management system need to be deployed first and
maintained continually through its lifecycle. To address
this problem, operator can use simple scripts which leverage
OS's local service (e.g. sshd) to ``manage'' a management
system centrally. But this approach is not scalable. Trying
to uses more advanced management systems which use their own
peers to manage another one will introduce the same problem
recursively.  \note{define managment loop here, and we
automate the loop within peers in SMON}

In this paper, we propose a Self-Managed Overlay Network
(SMON) that addresses the above problem. SMON is designed as
a distributed system with built-in self-management
capability, namely, self-deployment, self-recovery and
self-update. It consists of peers on every target machines
where applications will be deployed and managed. The peers
actively monitor each other and automatically deploy new
peers on fresh machines, recover failed peers, and update
peers of old versions. The collective behaviour of all peers
gives rise to a distributed system that can automatically
deploy itself to a set of machines, recover failed process
as soon as possbile and update itself to new versions
online.

\comment{
Running SMON peers will automatically deploy
and start new instances of SMON peers on machines where SMON
peers are not installed.  In this way, operator only needs
to deploy and start one SMON peer and SMON will be deployed
to all the target machines quickly. 

Each SMON peer has an associated version number and it is
stored persistently in configuration file. The version
number can be used to update SMON ``online'' to newer
versions.  When two peers communicate with each other and
find a difference in version, they work cooperately to
update the lower version to the latest verison. Using
epidemic communication, the whole SMON system will be
updated eventually once a single peer is updated.

SMON may be stopped because of machine failures or other
reasons. The failed peers will be detected and started 
by running ones automatically.
}


%SMON can deploy itself to a set of target machines
%automatically. While it is running, it monitors itself and
%recovers failed peers. If a new version of SMON is
%available, it will update itself online. SMON can also
%deploy and maintain a set of distributed applications. User
%can use a set of management interfaces to query status and
%set parameters of SMON and managed applications.

In designing SMON there are several challenges.

First, auto authentication while protecting user
credentials.

to deploy or start an installed SMON peer, a
SMON has to login into a remote machine and execute several
commands.

The second challenge is
scalability: epidemic

The third challenge is robust. epidemic, the control loop,
the monitor-reaction model.

The forth is extensible.


%It is of great challenge to manage distributed application
%when the scale is large. Distributed application management
%system is designed to ease the burdens of deploying and
%maintaining distributed applications in large-scale
%computing platforms. It provides user-friendly interfaces for
%people to deploy, configure, monitor and control distributed
%applications.

%To achieve
%these goals, a management system faces several challenges.
%It must have good scalability so as to deploy applications
%and broadcast control commands efficiently.  Host or network
%failures must be carefully handled to reduce their negative
%effects on application management to near minimum. When
%application is started, the management system monitors its
%running status. It must provide a extensible way for
%developers to define abnormal status of the application and
%the corresponding actions to peformance against abnormal
%status.

in summary, the paper makes following contributions.
scalable and robust self-management scheme using epidemic
alogorithm.

implementation on planetlab

evaluation showing scalability

the paper layout is.

\comment{
\note{what is involved in management activity}

\note{the problem in current management tools:}

\note{our solution}

\note{the challenges}

scalable, robust and simple (simple made robust possible)

and extensible?

\note{how we address these challenges}

\note{a short summary on BON status/results}

\note{contribution}

\note{paper layout}
}
% vim:foldmethod=marker:textwidth=60

