\section{Introduction}

Distrbuted systems are at the heart of today's Internet
services. While the cost of commodity computers continues to
drop, it is common for distributed computing platforms to
contain hundreds or thousands of computers, such as
PlanetLab, Amazon EC2 and Teragrid. These platforms can
provide huge amount of storage and computing capabilities
that boost performance of distributed applications
significantly. The applications must be carefully designed
to scale to thousands of computers and fully utilize the
resources.
%handle frequent and inevitable failures at host,
%application and network levels.

%As the scale of distributed applications become larger than
%ever, it is hard to manage them efficiently on distributed
%platforms.  Managing a distributed application involves
%several tasks.  The application must be first deployed to a
%set of computers. After the application is configured and
%started, its status need to be monitored for detecting and
%recovering from failures, ensuring sustained performance or
%just for collecting runtime data and statistics.
%Administrators may also want to send commands to control
%the application on demand.
Distributed application management system is designed to
simplify tasks involved in deploying and maintaining the
applications after they are designed and implemented. To
efficiently manage distributed applications at large scale,
the manangement system is designed in distributed approach.
It consists of many ``peers''\footnote{The word peer here
will refer to client if the management system is in
client/sever architecture.} on each machine where
application will be deployed. The peers form an overlay
network, and they can work cooperately to deploy application to
a set of machines.  After the applicaiton is started, each
peer can monitor and maintain application's processes within
the same machine box.  Administrator can send messages into
the overlay to query application status and execute control
commands on demand.

The distributed design makes management system a distributed
application in essential and it introduces an important
problem: the management system need to be deployed first and
maintained continually through its lifecycle. Currently,
this problem is addressed by using centralized approach. A
central controller first deploy and start the peers of
management system to a set of machines. After that, the
controller periodically monitors the peers and recover the
failed ones. If a crashed machine is replaced by a new one,
the controller has to deploy a new instance of peer on the
fresh machine. The controller can also upgrade peers to new
versions with bug fixes and improved performance. The
centralized approach has several disadvantages. First, it is
not scalable at deploying peers. Second, it is not efficient
at monitoring the peers. Frequent monitoring detects peers
failures quickly, but it puts heavy burden on network
resource of the central controller when establishing
connections and sending messages to remote machines. Third,
The static star toploy from central controller to peers'
machines cannot handle changes or failures in network
environments, such as network partition.
%What is more, trying to use more advanced management
%systems which use their own peers to manage another one
%will introduce the same problem recursively.  \note{define
%managment loop here, and we automate the loop within peers
%in SMON}

In this paper, we propose Self-Managed Overlay Network
(SMON) that addresses the problem of deployment and
maintainnance of distributed management systems. SMON is a
distributed management system with built-in self-management
capability, namely, self-deployment, self-recovery and
self-upgrade. It consists of peers on every target machines
where applications will be deployed and maintained. The
peers monitor each other and automatically deploy new peers
on fresh machines, recover failed peers, or upgrade peers of
old versions. The collective behaviour of all peers gives
rise to a distributed system that can deploy itself to a set
of machines, recover failed peers and update itself to new
versions. SMON can also manage a set of applications.
%especially another distributed application management
%system. In this way, SMON's management functionality can
%extended greatly.

\comment{
Running SMON peers will automatically deploy
and start new instances of SMON peers on machines where SMON
peers are not installed.  In this way, operator only needs
to deploy and start one SMON peer and SMON will be deployed
to all the target machines quickly. 

Each SMON peer has an associated version number and it is
stored persistently in configuration file. The version
number can be used to update SMON ``online'' to newer
versions.  When two peers communicate with each other and
find a difference in version, they work cooperately to
update the lower version to the latest verison. Using
epidemic communication, the whole SMON system will be
updated eventually once a single peer is updated.

SMON may be stopped because of machine failures or other
reasons. The failed peers will be detected and started 
by running ones automatically.
}


%SMON can deploy itself to a set of target machines
%automatically. While it is running, it monitors itself and
%recovers failed peers. If a new version of SMON is
%available, it will update itself online. SMON can also
%deploy and maintain a set of distributed applications. User
%can use a set of management interfaces to query status and
%set parameters of SMON and managed applications.

In designing SMON there are several challenges.

The first challenge is that SMON should have good
scalability. The second challenge is robustness. When SMON
is running, it should handle machine and network failures
automatically and gracefully \emph{without human
intervention}. Third, certain security guarantees should be
made to keep communication among peers authenticated and
confidential. This prevents occasional miss use of the
system, and also protects the credential used in deloying
new peers on fresh machines. The fourth challenge is that
SMON should have good extensibility.

SMON's design addresses the challenges as follows. For the
first challenge, SMON maintains an unstructured overlay and
uses epidemic algorithm to disseminate data among peers. The
epidemic algorithm ensures good scalability ($O(\log N)$)
when system goes large. For the second challenge. We should
first thank to epidemic algorithm because the random and
epidemic communication pattern makes it easy to handle
network partitions. And we further use a stateless
monitor-reaction model to describe and design SMON peer
behaviours. Because the model is simple and stateless, the
peers can runs automatically for long. For the third
challenge, we use a shared key to authenticate and encrypt
peers' communication. And a separate authentication agent is
used to store and protect login credentials. When a peer
needs to login into another machine and deploying a new
peer, it redirects the authentication challenge from remote
machine to the agent and replies the response solved by the
agent to the machine. The credential is never leaked out of
the agent. The authentication load on the agent is very
light and a single agent can support a number of peers. For
the forth challenge, SMON can be easily extented by upgrade
itself to a new version with new features or improved
performance. It also acts as a self-managed kernel and any
application upon SMON turns out to have self-managment
capablilty. Thus, more advanced management tools can be
built upon SMON and extends the management functionalities
greatly.

%It is of great challenge to manage distributed application
%when the scale is large. Distributed application management
%system is designed to ease the burdens of deploying and
%maintaining distributed applications in large-scale
%computing platforms. It provides user-friendly interfaces for
%people to deploy, configure, monitor and control distributed
%applications.

%To achieve
%these goals, a management system faces several challenges.
%It must have good scalability so as to deploy applications
%and broadcast control commands efficiently.  Host or network
%failures must be carefully handled to reduce their negative
%effects on application management to near minimum. When
%application is started, the management system monitors its
%running status. It must provide a extensible way for
%developers to define abnormal status of the application and
%the corresponding actions to peformance against abnormal
%status.

In summary, the paper makes following contributions.  We
design a scalable, robust and extensible self-management
system based on epidemic algorithm and monitor-reaction
model.  We implement it on Planet-Lab. The evaluation shows
that SMON has good performance and achies good scalability.

The rest of paper is organized as follows. We describe SMON
design in section~\ref{sec:design}. Section~\ref{sec:impl}
decribes our implementation on Planet-Lab platform. We
present evaluation of SMON in section~\ref{sec:eval}.
Section~\ref{sec:related} relates SMON with previous systems
and section~\ref{sec:conclusion} concludes.

\comment{
\note{what is involved in management activity}

\note{the problem in current management tools:}

\note{our solution}

\note{the challenges}

scalable, robust and simple (simple made robust possible)

and extensible?

\note{how we address these challenges}

\note{a short summary on BON status/results}

\note{contribution}

\note{paper layout}
}
% vim:foldmethod=marker:textwidth=60

