% vim:tw=60

\section{Introduction}
\label{sec:intro}

As industry is moving towards \textit{cloud computing}, many
players in the industry are building large and complex data
center applications on top of clusters of commodity PCs.
Ensuring the correctness and performance of the applications
are critical to providing sustained service to clients.
However, bugs have been challenging service developers and
maintainers continually.

However, the inherent complexity of the system hinders
people from understanding the system easily. The system are
usually built in a tiered fashion, with each tier providing
certain abstraction to the upper layer. Within a logical
tier, its function is implemented as a distributed system.
It may consist of hundreds or thousands of distributed
processes which works coorperately to fulfill the requests
from upper tier. Within a particular process, asynchronous
staged handling of messages is well adopted to fully utilize
the computing resource on a node. By leveraing the above
techniques, developers have built large scale of complex
systems to serve requests worldwide. The handling of a
request is splitted into pieces of tasks which are executed
distrbutedly across tiers, processes and stages. 

Understanding runtime behaviour of the complex system is key
to verify system design, debug its correctness and
performance problems. By tracking task pieces and their
causal dependencies, we can construct task flow by linking
together pieces of its execution throughout the system.  In
our previous work, we further developed techniques to
automatically tracking tasks and infer hierarchical task
models. Using the task models, developers can better
understand the structures of components and their
dependencies, and use debugging tools to instrument the
system and verify the behavior of tasks at appropriate
layers.

The production system contains abundant logging information
on system status, but it is not fully explored yet. It is
useful for two reason. First, the log is added by developers
who are familiar with the system.  The resulting log is a
faithful records of system runtime behaviour. Second, log
usually contains both state report and high level
descriptions. The derived task models is of value for both
automatic processing and human understanding.

The hierarchical structure of tasks is often consist with
the hierarchy of data processed by tasks. For example. In
cosmos, data are organized as streams and streams are
consisted of extents. By inspecting cosmos logs, task that
processing a stream is splitted into subtasks. The first
subtask do some stream level processing (open, etc.), and
the following tasks process extents in the stream. But the
task boundaries are not marked in the log.

By mining and leveraging data hierarchy, we can use the
information to infer the task hierarchy. A log item which
starts to process a stream marks the begin of a task. The
task lasts before the log item which process another stream.
Within the task, it is splitted into subtasks at
extent-processing boundary.

In this paper, we present a method for automatically infer
data hierarchy recorded in system log and use the
information to construct task models.  We also describe our
experiences in using the inferred task models to
understanding the system design and debug performance
problems.

\comment{
It faces several challenges. First, logs are not
written in well strucutured form. Second, log contains a lot
information other than data identifer, and the noises must
be reduce automatically with best effort.

}

\comment{
and build hierarchical task models to better
understand and check system runtime behaviour.

Meanwhile, debugging correctness and performance problems
for the systems are difficult.

System log is a rich source of information for understanding
system runtime behaviour.

Researchers have proposed several techniques, but system log
is not explored much.

In this paper, we try to build hierarchical task model
through logs.

Contribution/Highlights:

Goal: Understanding system behaviour by exploiting existing
system logs

magpie: more details on low level properties d3s: customized
log point pip: manual annotation

our advangtage: 

existing log more on high level semantic
}

\section{Design}
\label{sec:design}

In this section, we describe the detailed process of mining
data hierarchy using system log. It works in two phases.

\subsection{Extract key-value pairs}
\notes{value format are limited, we can find value part
easily, and track back in the text to find its key name.}

\subsection{Mining hierarchical relation of keys}

\notes{
\begin{itemize}
\item rule 1: keyS $>$ keyP, then keyS comes before keyP
\item rule 2: keyS $>$ keyP, then there're 1-to-many mapping relation between keyS and keyP value set.
\end{itemize}
}

\subsection{Construct hierarchical task models}
\notes{In this step, we summarize the key relations into
partial orders. and use it to construct task models.}

\section{Discussion}
\label{sec:discussion}
Discussion. \notes{topic need decided}

\section{Experiences}
\label{sec:exp}

\subsection{Understanding system behaviour}
\notes{understanding cosmos client, en and csm, but the
dependency line may be missing. Because log doesn't contain
enough information to join tasks among threads.}

\subsection{Guide on debugging}
\notes{
using the model to guide on debugging cosmos network
library.

}

\section{Related Works}
\label{sec:related}
Related Works

\section{Conclusion}
\label{sec:conclusion}
Conclusion

\section{References}
\label{sec:ref}
References

