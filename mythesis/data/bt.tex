% vim:textwidth=70
\chapter{分布式数据分发系统优化}
\label{chap:bt}

\section{本章引言}

%motivation: large file distribution in cooperative peers.
%problem: improve performance
% 说清楚一个应用场景,作为文章的backbone

数据分发技术是很多分布式应用的基础。P2P的文件共享系统将电影、歌曲和软
件CD等大文件分发给系统的每个用户。通过用户间合作下载,共享系统能够获得
几乎无限制的扩展性,同时,大大降低了数据源的负载。由文件共享系统产生的
数据传输,已经占了Internet流量的很大一部分。除此之外,人们还利用数据分
发技术,有效地将数据在CDN网络内部分发~\cite{fastreplica}。或者将这项技
术应用在分布式文件系统~\cite{sharkfs},让文件系统的用户安全地发现与下
载相互拥有的文件块,降低了服务器的压力。

数据分发的主要问题是将数据,尤其是大文件数据,分发到数量众多的用户或者
机器上。解决这个问题通常使用的基本技术是将文件分为逻辑上连续的若干块,
系统用户或者节点交换相互拥有的文件块,而不是直接向数据源服务器请求数据。
这样做的好处是显然的,首先降低了服务的压力,其次,这样的设计具有很好的
可扩展性,随着系统规模增大,总体的传输带宽也会增大。再次,文件块级别的
操作提供了更细粒度的负载均衡,从而系统节点能够动态的调整下载策略,绕开
可能的网络错误,以及获取更高下载性能。

进一步的,可以根据节点之间的通信拓扑将数据分发系统分为两种类型。swarm
类型的分发系统使用了随机连接拓扑,系统节点可以与任意其它节点通信。而
stream类型的系统使用了树、森林或者mesh等有约束结构的拓扑将节点连接起来。
swarm类型分发系统产生时间较stream类型系统更早。
BitTorrent~\cite{bittorrent}是swarm系统的代表。BitTorrent使用一个集中
式目录服务器记录系统节点与它们下载文件的进度。BitTorrent的客户端之间通
过这个目录服务器寻找可以提供文件块下载其他节点。目前,一些基于
BitTorrent协议的客户端也利用了DHT~\cite{kademlia}技术,实现了分布式结
构的目录服务器。基于stream方式的分发系统,包括
SplitStream~\cite{splitstream}和Bullet~\cite{bullet}等,试图通过使用精
巧的节点选择算法来解决负载均衡和传输效率问题,其结果是使系统节点构成了
树、森林或mesh等结构拓扑,并且在分发过程中使这一结构尽量稳定。这一方法
带来的好处是可能获得更高的聚集带宽,然而这需要节点之间的行为很好的同步,
因而对底层网络提出了更高的要求。在节点之间不能很好同步的情况下,stream
类型系统的传输效率会降低。

在Internet环境下基于swarm的分发技术得到了广泛应用。相比stream类型系统
,swarm更好的适应了Internet网络环境的动态变化,能够提供良好的传输性能
,同时也易于设计与实现。BitTorrent~\cite{bittorrent}文件共享协议是
swarm分发系统的典型代表。SharkFS~\cite{sharkfs}将swarm应用于分布式文件
系统,降低了文件服务器的负载。Coblitz~\cite{coblitz}将swarm应用于可控
的CDN网络,设计了基于HTTP协议的大文件分发下载服务。

% the point of paper
本文的研究内容始于一个基本的问题:如何提高swarm分发系统~\footnote{下文
中,分发系统都指基于swarm的分发系统}的性能?已有的工作从两方面指出了优
化的途径。\onlinecite{bns}通过模拟指出,使用临近邻居选择可以降低
BitTorrent的跨ISP流量,同时提高下载性能。然而~\onlinecite{bns}模拟使用
的网络拓扑过于简单,并不基于真实的网络数据。\onlinecite{taming,
zeroday}的工作使用了不同的技术,使临近邻居选择可以很容易的应用与部署到
实际分发系统。另一方面,\onlinecite{Bharambe2006}的工作显示了,固定上
传节点数目的带宽分配策略并不是最优的,动态的带宽分配策略可能会进一步提
高分发性能。

% 通过比较相关工作,引出本文的内容
% 说明两种优化方法都是已有的,但是没有详细的研究

虽然已有工作已经在提高分发系统性能方面做出了很多工作,然而还有很多问题
没有回答。具体来说,下面这些问题还没有回答清楚:

\begin{itemize}
  \item 临近邻居

  \item a

  \item a

  \item a

\end{itemize}

% contribution?
通过广泛的模拟测试,我们得到了如下结论:

1. bns

2. bwalloc

3. lrf的影响

4. bns+bwalloc

再对结论小小的总结一下。

本章接下来的内容安排如下。

%\section{或者:临近节点选择、带宽分配、文件块选择}

\section{优化方法}

本节首先给出分发系统的一般模型,并讨论了两种分发性能优化途径。

\subsection{分发系统模型}

一般来说,一个数据分发系统可以用如下模型描述。

一个分发系统包括一个目录服务器和一组参与数据分发的节点。目录服务器维护
着节点列表与节点下载文件的进度。节点通过向目录服务器发送请求,获知系统
内其它节点的状态,包括节点列表与节点下载进度。同时,节点也会向目录服务
器更新自己的下载进度。分发系统的节点通过相互交换文件块完成文件下载,达
到数据分发的目的。

% 如何说的足够general,又能让基于bt的eval有说服力?

对一个分发系统来说,邻居选择、上传带宽分配和文件块选择是决定系统性能的
关键设计点。邻居选择是指节点向谁请求数据。分发系统中的节点数目众多,节
点可以从许多其它节点那里请求数据,尝试向每一个系统节点发送数据请求是效
率低下的,需要从中选择其中的一些作为自己的邻居,并从邻居节点获取数据块。
一个数据分发系统中,节点在获取数据的同时,还要满足来自其它节点的数据请
求。由于请求数目可能很大,如何分配上传带宽也是需要考虑的设计内容。节点
请求数据的单位是文件块,在同时有多个文件块需要下载的情况下,选择哪一块
下载是文件块选择需要决定的内容。

BitTorrent使用了随机的邻居选择策略。中心目录服务器(tracker)返回一组
随机节点,BitTorrent的节点和邻居交换相互的数据块位图,得知对方已有的数
据块集合。BitTorrent通常使用LRF算法确定数据块的下载优先级,也就是会优
先下载所有邻居中副本最少的数据块。这样的方法能够很好的解决最后一块数据
块需要很长时间才能下载的问题。BitTorrent会优先给那些给自己上传过数据的
节点上传数据,并且对方上传带宽越大,优先级越高。采用这样的策略是因为
BitTorrent是为用户参与的P2P系统设计的,因而需要一定的激励机制,防止
free-rider的发生。已有的研究显示,BitTorrent的设计使得系统性能接近最优
解。

与BitTorrent不同,SharkFS使用了不同的邻居选择、上传带宽分配和文件块选
择策略。其本质是因为系统的应用场景不同。SharkFS实现了一个基于
DHT~\cite{coral}的分布式目录服务器。其邻居选择策略是选择最近的邻居,近
的定义是节点间延迟。SharkFS的节点对每一个需要下载的数据块,都会查询拥
有这个数据块的节点集合,并选择最近的节点请求数据。对于每个要下载的数据
块,SharkFS并不给出它们的下载顺序,因此是随机下载的。对于上传带宽分配
,SharkFS的文章中并未给出描述,我们可以认为是FIFO的。

其它一些分发系统,无论其应用场景、设计理念如何不同,都可以从邻居选择、
上传带宽分配和文件块选择这三个方面描述其分发协议的设计。

% sharkfs: nearest peer, random piece, peer selection unspecified
% bittorrent: random peer, random/LRF, tit-for-tat
% 

% design figure

%节点选择和数据块选择的重要性。

本文考虑了两种提高分发性能的方法:选择临近节点作为邻居列表和动态上传带
宽分配。

\subsection{临近邻居选择}

%邻居选择,平均下载时间分析

临近邻居选择是指节点选择从“近”的邻居那里下载文件块。近的度量可以是节
点间延迟、带宽等网络连接属性,也可以是以延迟、带宽和丢包率为自变量的函
数值。从应用角度来看,以延迟时间(RTT)作为网络距离的度量是简单而有效
的方法。通常来看,延迟低的节点之间网络连接情况优于延迟高的节点。

% xxx: 分发树?and a bit of math

选择临近节点作为邻居列表从直观上来看是十分自然的。从单个节点来说,从近
处的节点下载文件块的速度更快,从而得到整个文件的时间更短。从整个分发系
统来说,完成整个文件分发的时间取决于每个文件块的分发时间,而每个文件块
的分发都动态构成一棵树。如果使用文件块传输时间为树上的边赋值,选择临近
节点后,分发树上的每条边的值以一定概率降低,从而整个分发性能得到提高。

换个角度说,一个分发系统实际上使用了epidemic算法分发数据,可以将这一过
程建模为一个分支过程,系统规模是$n$。如果只考虑一个数据块的分发过程,
每一轮每个节点都随机给其他$f$个节点传输一个数据块,在$r$轮之后,拥有这
个数据块的节点占整个系统的比例$Y_r$是:
\begin{equation}
\label{equ:epidemic}
Y_r \approx \frac{1}{1+ne^{-fr}} 
\end{equation}

分发固定比例$Y_r$需要的轮数$r$和$n$成$\log$关系,也就是$r =
O(\log(n))$。在实际系统中,完成每一轮都是需要时间的,因而分发固定比例
$Y_r$需要的时间可以近似表示为:
\begin{equation*}
T = T_0 \times r
\end{equation*}

$T_0$是传输一个数据块需要的平均时间。相对随机节点选择,临近邻居选择降
低了$T_0$的值,因而从根本上降低了系统的分发性能。

临近节点选择方法,可以有两种实现方式。第一种方式需要节点自己判断并选择
临近邻居。节点从目录服务器那里获知其它节点列表,而目录服务器并不了解节
点之间的网络连接特性,通常只会返回一组随机节点列表。得到这组列表后,节
点之间可以通过邻居列表的节点进行探测得到网络连接的特性。第二种方式是,
目录服务器直接返回一组临近节点,而节点并不知道返回的邻居节点的特性。这
样的设计只需要改变目录服务器,而不需要更改每个节点的设计,因而在实际中
更容易部署。

\subsection{动态上传带宽分配}

% 先给出intuition,然后分析,最后实现方式

分发系统中的一个节点可能会收到多个来自其它节点的请求,由于请求数量可能
是很多的,不可能同时满足请求的文件块。那么,按照什么顺序满足谁的请求就
是需要解决的问题。BitTorrent选择给固定数目的k(通常是5)个节点上传数据,
并且每10秒重新选择上传目标节点~\footnote{未包括乐观unchoke的情况}。这
样的上传策略并不一定是性能最优的。

为了理解上传带宽分配的意义,我们首先考虑一个简单的例子。考虑一个三个节
点构成的分发系统,$v_1$, $v_2$, $v_3$。$v_1$的上传带宽是1,而$v_2$和
$v_3$上传带宽是2,它们的下载带宽无限制。初始时刻,$v_1$上有大小为1的数
据块。$v_1$可以将上传带宽平均分给$v_2$和$v_3$,从而在时刻2,$v_2$和
$v_3$都得到了数据块,整个系统完成了数据分发。另外一种上传带宽分配是
$v_1$将带宽全部分给$v_2$,在时刻1,$v_2$得到了整个数据块,同时给$v_3$
上传数据。在1.5时刻$v_3$也得到了数据块。从而整个系统在1.5时刻就完成了
数据分发。如果系统中包含其它节点,在时刻1,$v_1$就可以向其它节点传输数
据,分发的效率更加提高。
% graph

从根本上说,数据分发系统的全局目标是在最短时间内,让所有节点有得到分发
的文件。而分发的过程取决于每个节点的局部策略。从这一角度来说,每个节点
可以使用贪心策略通过局部最优使整个系统的性能接近最优。具体地,每个节点
的贪心策略应该是使每一个上传的数据块在最短的时间内传输完。形式上xxx

动态如何,静态如何,有什么问题。

实现方式

\ref{equ:epidemic}并不会。

\section{优化效果评测}

\subsection{评测方法}

为了了解临近邻居选择和动态带宽分配对分发系统性能的优化效果,回答本章引
言中提出的问题,我们进行了基于模拟的广泛测试。

\subsubsection{模拟器}

我们使用了基于BitTorrent的模拟器bitsim模拟200个节点的分发系统分发128MB
文件的行为,包括不使用优化算法和使用优化算法的情况。

模拟BitTorrent是因为BitTorrent是swarm类型分发系统的典型代表,它的行为
与性能得到了广泛的研究,其它swarm分发系统都采用了和BitTorrent相似的技
术。同时,BitTorrent的源代码也是公开的(版本5.0.9)之前,可以仔细分析
它的实现方式,实现尽可能相似的模拟结果。也可以在实际分布式计算平台上验
证模拟器是否和实际运行结果一致。而一些研究性系统或者商业系统由于无法实
际运行,因而难以设计实现相一致的模拟器。

bitsim模拟了标准的BitTorrent分发协议,包括choking/unchoking,乐观
choking,LRF块选择算法等。bitsim可以根据输入网络参数,模拟网络层的传输
延迟、TCP吞吐量和最大上传带宽。TCP的吞吐量依据流行的PFTK~\cite{pftk}模
型计算,模型中用到的窗口大小是$2^{16}$bytes。

我们的模拟使用了200个节点的规模。为了得到具有代表性的真实Internet网络
拓扑,我们使用了iPlane~\cite{iplane}的数据。iPlane是一个网络测量服务,
提供了Internet上精确的网络间丢包率和延迟测量值。iPlane定期的参与到若干
BitTorrent分发系统中,以测量节点的上传带宽。模拟使用的200个节点是从
iPlane测量的节点集合中随机采样出的。在模拟中,200个节点在短时间内同时
进入分发系统。

% 节点加入退出

% metric
% 实验内容(bias,bwalloc,lrf...)

\subsection{临近邻居选择}

% 现象、趋势、小结论、原因、前提条件,讨论配置变化的影响
% 每个subsection后给出一个总结
% 直接看图和表就能说明问题

我们首先测试了邻居节点选择对文件分发性能的影响。在测试中,节点每次向
tracker报告自己状态时,如果自身邻居节点数量少于设定值,会向tracker请求
返回系统内的若干随机节点(典型配置是50)最为自己的邻居。我们希望选择临
近的节点作为邻居会提高分发的性能,因而tracker返回的节点包括两部分:临
近节点和随机选择节点。但是这两部分节点的数量、比率在什么情况下会是最优
并不清楚。接下来我们试图通过实验测试的办法给出结论。

首先需要验证临近邻居节点选择是否会提升数据分发性能。我们设定临近比率是
80\%。也就是是在tracker返回的节点列表中,80\%是以网络延迟为距离定义,
离报告节点最近的一组节点,剩下20\%是从系统中随机选择的若干节点。这样做
的原因是避免由于全部邻居节点都按照最近距离选择,造成网络分割的现象。

\begin{figure}
  \centering
  \begin{minipage}{0.8\linewidth}
    \centering
    \includegraphics[width=1.0\linewidth]{bias80}
    \caption{临近邻居比率80\%时下载时间CDF曲线}
    \label{fig:bias80}
  \end{minipage}
\end{figure}

图~\ref{fig:bias80}是节点获取整个文件所用时间的累积分布函数(CDF),在
图中对比了使用随机邻居节点的结果和使用临近邻居节点选择后的结果。从
整体上来看,采用临近邻居节点选择后,曲线相比正常实现向左有明显移动。这
意味着相同数量的节点完成下载所需的时间更少了。为方便对比,图~
\ref{fig:bias80}上的关键统计数据总结在表~\ref{tbl:bias80}中。

可以看出,相比随机邻居选择,80\%比率的临近邻居选择对下载性能有了一定改
进。下载时间中值从1150秒下降到了1093秒,性能改进是5.0\%;90\%的节点完
成下载时间从1288秒下降到了1214秒,性能改进是5.7\%;下载时间的平均值从
1151秒下降到了1071秒,性能改进是7.0\%。

\begin{table}
\centering
\begin{minipage}{0.8\linewidth}
\centering
\caption{临近邻居比率80\%下载性能统计}
\label{tbl:bias80}
\begin{tabular}{lccc}

\toprule[1.5pt]
    & 中值 & 90\%完成 & 平均值\\
\midrule[1pt]
随机邻居选择  & 1150s & 1288s & 1151s\\
80\% 临近邻居 & 1093s & 1214s & 1071s\\
性能改进      & 5.0\% & 5.7\% & 7.0\%\\
\bottomrule[1.5pt]
\end{tabular}
\end{minipage}
\end{table}

这样的结果是符合我们的预期的。然而相对来说,优化效果并不一定是最优的。
为了找到更优的邻居选择策略,我们变化临近邻居选择的比率,并观察对下载性
能的影响。图~\ref{fig:biaschange}是临近邻居比率从20\%变化到100\%时,下
载时间的平均值和中值的变化情况。

\begin{figure}
  \centering
  \begin{minipage}{0.8\linewidth}
    \centering
    \includegraphics[width=1.0\linewidth]{biaschange}
    \caption{下载时间中值与平均值随临近邻居比率变化图}
    \label{fig:biaschange}
  \end{minipage}
\end{figure}

可以看出,随着节点选择更多的临近邻居,系统的整体性能呈现单调增长的趋势。
尤其需要注意的是,在100\%使用临近节点时,系统的性能达到了最优,其下载
时间CDF曲线如图~\ref{fig:bias10}所示。相比随机节点选择,分发进度的中值
从1150秒减少到了936秒,性能提高了18.6\%,90\%进度的时间从1288秒减少到
了1088秒,性能提高了15.5\%,分发时间的平均值1151秒减少到了916秒,性能
提高了20.4\%。

\begin{figure}
  \centering
  \begin{minipage}{0.8\linewidth}
    \centering
    \includegraphics[width=1.0\linewidth]{bias10}
    \caption{临近邻居比率100\%时下载时间CDF曲线}
    \label{fig:bias10}
  \end{minipage}
\end{figure}

\begin{table}
\centering
\begin{minipage}{0.8\linewidth}
\centering
\caption{临近邻居比率100\%下载性能统计}
\label{tbl:bias10}
\begin{tabular}{lccc}

\toprule[1.5pt]
              & 中值 & 90\%完成 & 平均值\\
\midrule[1pt]
随机邻居选择  & 1150s & 1288s & 1151s\\
100\% 临近邻居 & 936 & 1088s & 916s\\
性能改进      & 18.6\% & 15.5\% & 20.4\%\\
\bottomrule[1.5pt]
\end{tabular}
\end{minipage}
\end{table}

在使用100\%临近邻居时,没有发生担心的网络分割情况。这是因为节点的邻居
集合的度为50,其数量足够大,保证了很强的网络连通性。这样的结果告诉我们,
在实际分发系统设计中,可以使邻居节点包含尽可能多的临近节点,使分发性能
得到最大程度的提高。如果担心网络分割发生,只需要保留少数个随机选择的节
点就可以了。

%为了更仔细的研究系统特点,我们也收集了系统在分发文件过程中,节点的邻
%居分布

% 和系统总吞吐量的变化。

%图~\ref{fig:bias10path}
%
%\begin{figure}
%  \centering
%  \begin{minipage}{0.8\linewidth}
%    \centering
%    \includegraphics[width=1.0\linewidth]{ph}
%    \caption{临近邻居延迟分布}
%    \label{fig:bias10path}
%  \end{minipage}
%\end{figure}

\textbf{结论}:通过测试我们发现,临近邻居选择可以提高分发系统的性能。
同时,随着临近节点占节点邻居列表的比例越来越高,系统的分发性能也单调提
高,在邻居列表完全使用临近节点时达到最优。在实际应用中,为了防止网络分
割的发生,可以在邻居列表中包括少量常数个随机节点。

% path latency
% throughput

%1. 80\% bias: download time distribution, neighbour latency(需要重新测
%的), download rate from neighbours(可以从下载统计推导出)
%2. 不同bias
%3. 只保留若干长连接的bias

\subsection{带宽分配优化}

% 模拟中,上传带宽分配优化是如何实现的

使用上传带宽分配优化后,相比未优化时,系统性能有了很大提高,节点传输时
间的累积分布函数如图~\ref{fig:bwalloc}所示。图中右边的曲线是未优化时的
传输时间分布,左边的曲线是使用上传带宽分配优化后的时间分布,相比未优化
的情况,有了明显的性能改进,直观的说,传输时间曲线向左有了明显移动。

从统计数据上看(表~\ref{tbl:bwalloc}),未优化时传输时间的中值是1150秒
,90\%完成时间是1288秒,传输时间的平均值是1151秒,而使用上传带宽优化后
,传输时间的中值减少到了782秒,性能提高了32.0\%,90\%完成传输时间减少
到了926秒,性能提高了28.1\%,传输时间的平均值减少到了780秒,性能提高了
32.2\%。

\begin{figure}
  \centering
  \begin{minipage}{0.8\linewidth}
    \centering
    \includegraphics[width=1.0\linewidth]{bwalloc}
    \caption{上传带宽分配优化的下载时间CDF曲线}
    \label{fig:bwalloc}
  \end{minipage}
\end{figure}

\begin{table}
\centering
\begin{minipage}{0.8\linewidth}
\centering
\caption{上传带宽优化传输性能统计}
\label{tbl:bwalloc}
\begin{tabular}{lccc}

\toprule[1.5pt]
              & 中值 & 90\%完成 & 平均值\\
\midrule[1pt]
未优化    & 1150s & 1288s & 1151s\\
优化后    & 782   & 926s  & 780s\\
性能改进  & 32.0\% & 28.1\% & 32.2\%\\
\bottomrule[1.5pt]
\end{tabular}
\end{minipage}
\end{table}


%2. 提高了整体throughput, (bwalloc\_throughput)

上传带宽分配方法降低了数据块从一个节点传输到另外一个节点的时间,从整体
上看,提高了系统的聚集吞吐量。图~\ref{fig:bwalloc_throughput}比较了xxx。

\begin{figure}
  \centering
  \begin{minipage}{0.8\linewidth}
    \centering
    \includegraphics[width=1.0\linewidth]{ph}
    \caption{}
    \label{fig:bwalloc_throughput}
  \end{minipage}
\end{figure}

%3. 与half,double对比 (bwalloc\_cmp)

使用动态带宽分配后,传输时间是最快的,期望xxx是nearly optimal

图~\ref{fig:bwalloc_cmp}

\begin{figure}
  \centering
  \begin{minipage}{0.8\linewidth}
    \centering
    \includegraphics[width=1.0\linewidth]{bwalloc_cmp}
    \caption{不同上传带宽分配方法比较}
    \label{fig:bwalloc_cmp}
  \end{minipage}
\end{figure}

4. 上传节点数量分布 (bwalloc\_upnode)
图~\ref{fig:bwalloc_upnode}

\begin{figure}
  \centering
  \begin{minipage}{0.8\linewidth}
    \centering
    \includegraphics[width=1.0\linewidth]{ph}
    \caption{}
    \label{fig:bwalloc_upnode}
  \end{minipage}
\end{figure}

5. 原因:因为这个方法会选择上传的节点,和bias用latency选择类似。只不过
tracker没有帮助选择节点。比bias更好是因为分发truck的速度更快,考虑了自
身上传带宽的因素。

\subsection{lrf的影响}

1. 对bias的影响 (bias\_lrfvsrand)
图~\ref{fig:bias_lrfvsrand}

\begin{figure}
  \centering
  \begin{minipage}{0.8\linewidth}
    \centering
    \includegraphics[width=1.0\linewidth]{bias_lrfvsrand}
    \caption{}
    \label{fig:bias_lrfvsrand}
  \end{minipage}
\end{figure}

2. 对带宽分配的影响 (bwalloc\_lrfvsrand)
图~\ref{fig:bwalloc_lrfvsrand}

\begin{figure}
  \centering
  \begin{minipage}{0.8\linewidth}
    \centering
    \includegraphics[width=1.0\linewidth]{bwalloc_lrfvsrand}
    \caption{}
    \label{fig:bwalloc_lrfvsrand}
  \end{minipage}
\end{figure}

3. diversity
图~\ref{fig:diversity_bias}

\begin{figure}
  \centering
  \begin{minipage}{0.8\linewidth}
    \centering
    \includegraphics[width=1.0\linewidth]{ph}
    \caption{}
    \label{fig:diversity_bias}
  \end{minipage}
\end{figure}

图~\ref{fig:diversity_bwalloc}
\begin{figure}
  \centering
  \begin{minipage}{0.8\linewidth}
    \centering
    \includegraphics[width=1.0\linewidth]{ph}
    \caption{}
    \label{fig:diversity_bwalloc}
  \end{minipage}
\end{figure}

3. 原因: a) lrf会增进性能 b)对带宽分配优化影响相对小: bias以后,相近的
节点会有聚类效应,lrf能够很快的把类外的新的trunk拉回来。如果用random,
则很可能得到的trunk是类内的,分发数据的效率受影响。


\subsection{bias+带宽分配}
1. vs normal
图~\ref{fig:biasandbwalloc}

\begin{figure}
  \centering
  \begin{minipage}{0.8\linewidth}
    \centering
    \includegraphics[width=1.0\linewidth]{biasandbwalloc}
    \caption{}
    \label{fig:biasandbwalloc}
  \end{minipage}
\end{figure}

2. 不具有叠加效果: xxx: bias以后,节点就已经比较近了,相互之间传输带宽
比较好,再使用带宽优化改进不大。

\section{实际应用的考量}

实现方式,优化效果

ISP的带宽分配、节流策略.

\section{结论}
