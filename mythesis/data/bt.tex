% vim:textwidth=70
\chapter{分布式系统快速文件分发方法}
\label{chap:bt}

\section{本章引言}

motivation: large file distribution.

分布式系统上的文件分发技术在很多应用中都会被使用,例如软件发布~
\cite{xxx}、内容分发网络~\cite{xxx}和分布式文件系统~\cite{xxx}等。

在这些应用中,文件分发技术被用来将大文件分发到很多不同的机器上去。大致
上可以按照机器间拓扑的种类,将文件分发分为swarm和streaming两种。

本文研究了

swarm-like? cooperative caching? 

bt, fastreplica, coblitz


\section{大文件分发概述}


%\section{或者:临近节点选择、带宽分配、文件块选择}

\section{优化方法概述}

128M

\section{优化方法评测}

\subsection{评测方法}
1. cluster来做neighbour selection
2. 增加慢节点的优先级

可行的节点测量方法:existing measurement infrastructure(iplane, p4p), ad hoc
landmark

\subsection{biased neighbour}

% 现象、趋势、小结论、原因、前提条件
% 每个subsection后给出一个总结
% 直接看图和表就能说明问题

我们首先测试了邻居节点选择优化对文件分发性能的影响。在测试中,节点每次
向tracker报告自己状态时,tracker返回分发系统内的若干节点,典型配置下是
50。这些节点中的80\%是以网络延迟为距离定义,离报告节点最近的一组节点,
剩下20\%是从系统中随机选择的若干节点。这样做的原因是避免由于全部邻居节
点都按照最近距离选择,造成网络分割的现象。

\begin{figure}
  \centering
  \begin{minipage}{0.8\linewidth}
    \centering
    \includegraphics[width=1.0\linewidth]{ph}
    \caption{bias80}
    \label{fig:bias80}
  \end{minipage}
\end{figure}

图~\ref{fig:bias80}是节点完整下载整个文件所用时间的累积分布函数(CDF),
在图中分别显示了正常的BitTorrent实现的结果和采用邻居节点优化后的结果。
从整体上来看,采用邻居节点选择优化后,曲线相比正常实现向左有明显移动。
这意味着相同数量的节点完成下载所需的时间更少了。为方便对比,图
~\ref{fig:bias80}上一些重要点的数据总结在表~\ref{tbl:bias80}中。可以看
出,

\begin{table}
\centering
\begin{minipage}{0.8\linewidth}
\centering
\caption{bias80}
\label{tbl:bias80}
\begin{tabular}{ccc}

\toprule[1.5pt]
 header \\
\midrule[1pt]
body \\
\bottomrule[1.5pt]
\end{tabular}
\end{minipage}
\end{table}

这样的结果是符合我们的预期的。


1. 80\% bias: download time distribution, neighbour latency(需要重新测
的), download rate from neighbours(可以从下载统计推导出)
2. 不同bias
3. 只保留若干长连接的bias

\subsection{带宽分配}
1. vs normal
2. 提高了整体throughput, 
3. 下载一个truck时间不好测,可以看从邻居下载速度是否增加
4. 原因:因为这个方法会选择上传的节点,和bias用latency选择类似。只不过
tracker没有帮助选择节点。比bias更好是因为分发truck的速度更快,考虑了自
身上传带宽的因素。

\subsection{lrf的影响}
1. 对bias的影响
2. 对带宽分配的影响
3. 原因: a) lrf会增进性能 b)对带宽分配优化影响相对小: bias以后,相近的
节点会有聚类效应,lrf能够很快的把类外的新的trunk拉回来。如果用random,
则很可能得到的trunk是类内的,分发数据的效率受影响。


\subsection{seed的影响}
1. 还没做

\subsection{bias+带宽分配}
1. vs normal
2. 不具有叠加效果: xxx

\section{实际应用的考量}
实现方式,优化效果

ISP的带宽分配、节流策略.

\section{相关工作对比}

1. 用bias优化bt:已有工作只是选择了一种bias方法使用,重点在如何实现
bias,例如用cdn,用landmark,但是我们系统评测了可能的bias方法,并给出
了最优结论:选择k=1

2. 带宽选择

3. lrf, seed对优化方法的影响

4. 两种优化方法不具有叠加效果。

5. 不同系统应用优化方法的建议

\section{结论}
