\begin{resume}

  \resumeitem{个人简历}

  1981年7月4日出生于宁夏回族自治区银川市。

  1999年9月考入清华大学计算机科学与技术系,2003年7月本科毕业并获得工学学士学位。

  2003年9月免试进入清华大学计算机科学与技术系攻读计算机体系结构博士至今。

  \resumeitem{发表的学术论文} % 发表的和录用的合在一起

  \begin{enumerate}[{[}1{]}]

  \item \textbf{高崇南},余宏亮,郑纬民。可自举的分布式应用管理覆盖网络。
  中国计算机网络安全应急年会信息内容安全分会,2008。

  \item \textbf{Chongnan Gao}, Jing Sun, Jinfeng Hu, Ning Ning, Weimin
  Zheng.  ImDeploy: A Tool for Global Scale Service Deployment on
  Peer-to-Peer Networks, In Proceedings of The First International Workshop
  on Mobility in Peer-to-peer Systems (in conjunction with ICDCS), 2005.

  \item \textbf{高崇南},余宏亮,郑纬民。可自管理的分布式应用管理覆盖网络。
  《清华学报》。(已录用)

  \item \textbf{Chongnan Gao}, Hongliang Yu, Weimin Zheng. Towards An
  Self-managed Tool For Distributed Application Management. Science in
  China. (in submission)

  \item \textbf{高崇南},余宏亮,郑纬民。基于日志的系统任务模型推断工具及其
  应用。《计算机研究与发展》。(已投稿)

  \item Haohui Mai, \textbf{Chongnan Gao}, Xuezheng Liu, Xi Wang, Geoffrey
  M.\ Voelker. Towards Automatic Inference of Task Hierarchies in Complex
  Systems, In Hot Topics in System Dependability (in conjunction with OSDI),
  2008.

  \item Ning Ning, Dongsheng Wang, Yongquan Ma, Jinfeng Hu, Jing Sun,
  \textbf{Chongnan Gao}, Weiming Zheng. Genius: Peer-to-Peer Location-Aware
  Gossip Using Network Coordinates, In Proceedings of International Workshop
  on Grid Computing Security and Resource Management (in conjunction with
  ICCS), 2005. (SCI索引号:8623210)

  \item Youhui Zhang, Dongsheng Wang, \textbf{Chongnan Gao}, Weimin Zheng. A
  JDO Storage Cluster Based on Object Devices. In Grid and Cooperative
  Computing Workshops (GCC), 2004. (SCI索引号:8337165)

  \end{enumerate}

  \resumeitem{参与研究项目}
  \begin{enumerate}[{[}1{]}]

  \item 国家自然科学基金:对等计算及广域网虚拟平台

  \item 中国下一代互联网项目:弹性互联网络
  \end{enumerate}

  \resumeitem{研究成果} % 有就写,没有就删除
  \begin{enumerate}[{[}1{]}]

  \item 高崇南,孙竞。分布式系统信息监测软件:中国,2007SRBJ2932
  (软件著作权登记号)

  \end{enumerate}
\end{resume}
