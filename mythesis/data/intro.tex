% vim:textwidth=70
\chapter{引言}
\label{chap:intro}

\section{选题背景}
% 管理和任务模型怎么捏到一起

% 分布式系统需要管理->管理=部署+维护(=监测+恢复)

% SMON维护应用的方法(online、offline、ignore)只是一种维护机制,可以
% 用于跑实验或长期服务的维护。维护机制可以有其他选择,例如可以没有(就
% 是ignore),可以用scalpel动态抓任务模型,看有没有正确性或者性能问题。

% 维护=监测+动作
% 监测:进程死活,应用内部状态等待
% 动作:重启?获取任务模型等等。

Internet服务正在改变着人们的生活方式。搜索引擎改变了已有知识的组织与访
问方式,它使得全世界的人们,可以通过互联网方便地获取感兴趣的内容,这包
括在线文档、图片、视频等内容,也包括更专业的文献、新闻、金融信息、博客
等搜索服务。除了搜索引擎,在线邮件服务让人们通过PC或者手机就可以随时随
地处理自己的邮件。在线文档服务使得人们不必安装专业软件,就可以创建、编
辑与共享信息。同时,多个个体可以协同编辑同一文档,而不必考虑底层的存储
维护、一致性保证等工作。在线交友服务让人们有了新的互相认识与交流的方式,
它扩大了人们相互认识的途径,也改变了老朋友们维持友情的方式。

%其它internet服务还有在线视频、照片。在线视频,交友,照片,博客。影响了
%其它产业,例如手机

支撑这些internet服务的是一些大规模的分布式系统。分布式存储系统可靠的保
存着回答搜索请求所需要的巨大容量的信息,例如网页、图片、视频、地理信息
、金融数据等。从而保证了在任何时间、任何地点的人们,都可以有效地查询信
息。内容分发网络(CDN,Content Distribution Network)将网络服务“推送”
到离用户最近的网络边缘,这样,用户就能够就近快速访问服务,同时减轻了服
务提供者的负载。基于P2P覆盖网络的系统被广泛的应用在文件分发、共享,视
频组播等服务。P2P覆盖网络让用户能够相互共享与分发数据,从而有效提高了
网络资源的利用率,同时减轻了服务器的压力。在学术界与工业界,人们也在尝
试使用P2P技术,搭建具有高度可扩展性和稳定性的存储系统,提供“云存储”
(Cloud Storage)服务。

早期的internet服务基于单机系统。世界上最早的搜索引擎Archie起初只使用了
一台机器,它定期从一组FTP服务器获取文件列表,为用户提供文件查询服务。
随着用户越来越多,Archie逐渐从单机系统,演变成由前端和后端组成,在
internet上有多处服务器的分布式系统。

随着internet服务的普及与流行,支撑它们的分布式系统规模变的越来越大。以
Google为例,2006年的数据\footnote{Google有意保密其系统规模等参数,因此
难以获得最新数据}表明,支撑Google服务的集群规模已经达到了450000台机器,
其耗电量达到了20兆瓦特,平均每月电费在200万美元的规模。存储在这些集群
上的数据容量也很大,在2008年召开的Google I/O会议上,Google Fellow
Jeffrey Dean披露说,最大的BigTable运行实例包含了超过6 petabytes的数据。

% 现在,cluster规模: 450,000, 2006,how google works, wiki:google
% platform

有若干因素使得internet服务必然选择分布式系统作为关键组成部分。首先,
internet服务的成功,吸引了众多使用者。服务不再可能使用单机或者小规模集
群就能够胜任。随着普通计算机元件造价越来越低,人们可以很容易的搭建出规
模上千的分布式系统。用户请求被分散到不同的机器并行处理,提高了服务的吞
吐率与处理延迟。服务还会被分布到不同地区,使用户能够就近访问,减小了服
务的处理延迟。

其次,是为了保证服务的可靠性与可用性。用户希望服务总是可用的,也就是服
务总是可以访问的。同时,服务也应该是可靠的,用户的数据不应该丢失。因此,
系统使用冗余机制,将数据与服务复制多份。即便遇到故障,仍然能够处理用户
请求。

最后,有些internet应用,其本身就必须是分布式系统。例如,基于P2P的文件
共享,视频组播等。只有分布式系统,能够将世界各地的用户连接到一起。

\note{上面的内容还可以再扯一扯}

随着分布式系统被广泛应用,它也迅速成为了工业界与学术界的关注热点。各种
各样的分布式系统被提出并研究。我们注意到,分布式系统不仅难于设计,同时,
如何有效地管理已经设计与实现的分布式系统,如何调试分布式系统的正确性与
性能缺陷,也是非常重要的问题。

与传统单机软件不同,分布式系统需要被有效地管理。分布式系统由运行在不同
机器上的应用节点组成,应用节点通过网络相互连接,共同完成期望的功能。因
此,需要首先将系统部署到一组机器上才能运行。这其中的核心问题是如何将应
用分发到规模成百上千台机器上去,考虑到系统规模巨大,同时各种故障随着规
模增大而频繁发生(例如机器故障、网络连接断开),因此部署一个分布式应用
并不容易。在应用部署并启动之后,它的运行状态需要被紧密监测,以保证系统
运行正确、稳定,性能达标。

分布式系统的调试也是个挑战。规模大,并行处理,请求经过不同节点,异步处
理,更复杂的逻辑,错误处理。(或者应该说调试研究什么,而不是研究的难
点。)

本文分别研究了分布式系统管理与调试中的关键问题。具体的,针对xxx我们研
究了xxx。

% 从历史看分布式系统发展的脉络,从应用看分布式系统使用的必然性

% 随着系统规模越来越大,设计一个分布式系统遇到了新的问题与挑战。


% 与单机运行软件不同,一个分布式系统在设计与实现之后,如何管理与调试也面
% 临着诸多挑战。

% 1. motivate 管理和调试
% 2. 管理和调试捏到一起
% 3. current limitation

\subsection{Internet服务与分布式系统}

这章或许可以不要,等写完了下面章节的内容再回来看看。

这章其实就是扯淡,主要述说了internet服务中的分布式系统都有什么,各有什
么特点。

\subsection{分布式系统设计}

设计一个分布式系统需要解决许多困难问题,通常下面这问题是必须考虑的:

可扩展性;可靠性;容错;一致性;可扩展性

\section{本文研究内容}

\subsection{分布式系统管理}

部署(一边说问题,一边说相关研究)

监测

控制


\subsection{分布式系统调试}

分布式系统设计复杂,是为了满足可扩展性等要求。对调试有了新的困难。

基于对分布式系统运行状态检测的分析。


\section{本文研究贡献}

\subsection{研究贡献}

\subsection{目标使用者}

\subsection{内容组织}

