% vim:textwidth=70
\chapter{引言}
\label{chap:intro}

\section{选题背景与意义}
% 管理和任务模型怎么捏到一起

% 分布式系统需要管理->管理=部署+维护(=监测+恢复)

% SMON维护应用的方法(online、offline、ignore)只是一种维护机制,可以
% 用于跑实验或长期服务的维护。维护机制可以有其他选择,例如可以没有(就
% 是ignore),可以用scalpel动态抓任务模型,看有没有正确性或者性能问题。

% 维护=监测+动作
% 监测:进程死活,应用内部状态等待
% 动作:重启?获取任务模型等等。

Internet服务正在改变着人们的生活方式。搜索引擎改变了已有知识的组织与访
问方式,它使得全世界的人们,可以通过互联网方便地获取感兴趣的内容,这包
括在线文档、图片、视频等内容,也包括更专业的文献、新闻、金融信息、博客
等搜索服务。除了搜索引擎,在线邮件服务让人们通过PC或者手机就可以随时随
地处理自己的邮件。在线文档服务使得人们不必安装专业软件,就可以创建、编
辑与共享信息。同时,多个个体可以协同编辑同一文档,而不必考虑底层的存储
维护、一致性保证等工作。在线交友服务让人们有了新的互相认识与交流的方式,
它扩大了人们相互认识的途径,也改变了老朋友们维持友情的方式。

%其它internet服务还有在线视频、照片。在线视频,交友,照片,博客。影响了
%其它产业,例如手机

支撑这些internet服务的是一些大规模的分布式系统。分布式存储系统可靠的保
存着回答搜索请求所需要的巨大容量的信息,例如网页、图片、视频、地理信息
、金融数据等。从而保证了在任何时间、任何地点的人们,都可以有效地查询信
息。内容分发网络(CDN,Content Distribution Network)将网络服务“推送”
到离用户最近的网络边缘,这样,用户就能够就近快速访问服务,同时减轻了服
务提供者的负载。基于P2P覆盖网络的系统被广泛的应用在文件分发、共享,视
频组播等服务。P2P覆盖网络让用户能够相互共享与分发数据,从而有效提高了
网络资源的利用率,同时减轻了服务器的压力。在学术界与工业界,人们也在尝
试使用P2P技术,搭建具有高度可扩展性和稳定性的存储系统,提供“云存储”
(Cloud Storage)服务。

早期的internet服务基于单机系统。世界上最早的搜索引擎Archie起初只使用了
一台机器,它定期从一组FTP服务器获取文件列表,为用户提供文件查询服务。
随着用户越来越多,Archie逐渐从单机系统,演变成由前端和后端组成,在
internet上有多处服务器的分布式系统。

随着internet服务的普及与流行,支撑它们的分布式系统规模变的越来越大。以
Google为例,2006年的数据\footnote{Google有意保密其系统规模等参数,因此
难以获得最新数据}表明,支撑Google服务的集群规模已经达到了450000台机器,
其耗电量达到了20兆瓦特,平均每月电费在200万美元的规模。存储在这些集群
上的数据容量也很大,在2008年召开的Google I/O会议上,Google Fellow
Jeffrey Dean披露说,最大的BigTable运行实例包含了超过6 petabytes的数据。

% 现在,cluster规模: 450,000, 2006,how google works, wiki:google
% platform

有若干因素使得internet服务必然选择分布式系统作为关键组成部分。首先,
internet服务的成功,吸引了众多使用者。服务不再可能使用单机或者小规模集
群就能够胜任。随着普通计算机元件造价越来越低,人们可以很容易的搭建出规
模上千的分布式系统。用户请求被分散到不同的机器并行处理,提高了服务的吞
吐率与处理延迟。服务还会被分布到不同地区,使用户能够就近访问,减小了服
务的处理延迟。

其次,是为了保证服务的可靠性与可用性。用户希望服务总是可用的,也就是服
务总是可以访问的。同时,服务也应该是可靠的,用户的数据不应该丢失。因此,
系统使用冗余机制,将数据与服务复制多份。即便遇到故障,仍然能够处理用户
请求。

最后,有些internet应用,其本身就必须是分布式系统。例如,基于P2P的文件
共享,视频组播等。只有分布式系统,能够将世界各地的用户连接到一起。

%\note{上面的内容还可以再扯一扯}

随着分布式系统被广泛应用,它也迅速成为了工业界与学术界的关注热点。针对
各种应用的分布式系统被提出并研究。有xxx

我们注意到,分布式系统不仅难于设计,
同时,如何有效地管理分布式系统,如何调试分布式系统的正确性与性能缺陷,
也是非常重要的问题。

与传统单机软件不同,分布式系统需要被有效地管理。分布式系统由运行在不同
机器上的应用节点组成,应用节点通过网络相互连接,共同完成设计的功能。因
此,需要首先将系统部署到一组机器上才能运行。这其中的核心问题是如何将应
用分发到规模成百上千台机器上去,考虑到系统规模巨大,同时各种故障随着规
模增大而频繁发生(例如机器故障、网络连接断开),因此部署一个分布式应用
并不容易。在应用部署并启动之后,它的运行状态需要被紧密监测,以保证系统
运行正确、稳定,性能达标。

分布式系统的调试也是个挑战。规模大,并行处理,请求经过不同节点,异步处
理,更复杂的逻辑,错误处理。(或者应该说调试研究什么,而不是研究的难
点。)

本文分别研究了分布式系统管理与调试中的关键问题。具体的,针对xxx我们研
究了xxx。

% 从历史看分布式系统发展的脉络,从应用看分布式系统使用的必然性

% 随着系统规模越来越大,设计一个分布式系统遇到了新的问题与挑战。


% 与单机运行软件不同,一个分布式系统在设计与实现之后,如何管理与调试也面
% 临着诸多挑战。

% 1. motivate 管理和调试
% 2. 管理和调试捏到一起
% 3. current limitation

% \subsection{Internet服务与分布式系统}
% 
% 这章或许可以不要,等写完了下面章节的内容再回来看看。
% 
% 这章其实就是扯淡,主要述说了internet服务中的分布式系统都有什么,各有什
% 么特点。
% 
% \subsection{分布式系统设计}
% 
% 设计一个分布式系统需要解决许多困难问题,通常下面这问题是必须考虑的:
% 
% 可扩展性;可靠性;容错;一致性;可扩展性

\section{分布式系统管理}

分布式系统或分布式应用\footnote{在后面的叙述中,我们交叉使用“分布式系
统”或“分布式应用”来表达相同的意思。}管理包括若干问题。分布式系统需
要运行在由网络连接的一组机器上,不同的系统对机器、网络资源配置的要求也
不同。我们需要在分布式计算平台提供的资源基础上,选择出一个符合要求的子
集,这就是\emph{资源发现}。其次,需要将分布式系统\emph{部署}到这组机器
上,包括分发系统的安装程序,在每个机器上配置运行参数,并在各个机器启动
系统运行。在系统运行期间,需要\emph{监测}系统运行状态。这包括对系统所
在计算平台的监测,也包括对系统本身状态的监测。最后,管理者希望能够动态
的\emph{控制}系统的运行,包括恢复失败的系统进程等。

\subsubsection*{资源发现}

支撑分布式系统运行的分布式计算平台规模可能非常大,Planet-Lab包括了超过
900个节点,而Google使用的集群规模达到了数万台机器。通常一个分布式系统
并不需要使用计算平台上的所有机器。开发者或者管理者需要找到满足系统运行
的一组资源,包括一组机器以及连接这些机器的网络。

不同的分布式系统对资源的要求不同。科学计算程序希望能够找到空闲CPU、内
存都很大的机器,如果计算过程中有频繁的数据交换,那么这些机器之间的网络
连接状况也应该很好。一些科学计算程序运行时间较短,从几小时到几天。因此
它们更关心资源的当前使用情况。还有一些分布式系统提供长期的服务,例如资
源发现服务本身。它们定期收集计算平台的资源使用情况,并执行来自用户的查
询。这些服务并不需要使用很多CPU与内存资源,但是它们希望运行在稳定性高
的一组机器上,频繁的机器或网络故障会影响服务的质量。

资源发现服务接受用户对所期望资源的描述,并返回满足要求的一组资源,如果
没有资源能够满足用户的期望,则返回空。抽象来说,这是一个图的匹配问题。
资源发现服务定期的收集整个分布式计算平台的资源使用数据,包括每个机器的
计算资源和存储资源,以及机器之间网络连接情况,将其以适合的方式保存。用
户将对期望资源的描述用图的形式表达。图的节点描述了机器应当满足的条件,
图的边描述了机器之间的网络连接应当满足的条件。资源发现服务要在整个计算
平台这个大图上,寻找满足用户条件子图,这等价于k-clique问题,是一个
NP-hard问题。

\subsubsection*{部署}

部署是将分布式系统分发、安装到选定的一组机器上,并配置、启动系统运行的
过程。其核心是如何将系统分发复制到一组机器上。

可以使用集中式的方法部署分布式应用。管理者从一个集中控制机器,将系统的
安装程序远程复制到各个机器上,并逐个配置并启动应用的进程。这种方法最大
的优点是设计实现简单,许多情况下,只需要编写几个脚本就可以完成绝大多数
工作。

集中式的方法对小规模应用很有效,但是却不能很好的胜任大规模时的情况。首
先它的可扩展性很差,在部署多台机器时,所有的数据都通过本地网络分发,效
率低下。其次,在大规模情况下,网络与机器故障成为频繁发生的事情。集中式
方法使用的静态星型拓扑结构不能很好的处理这些故障。

因此,各种基于P2P算法的部署工具被研究并开发。这些工具本身也是一个分布
式系统。工具的节点分布式在各个机器,节点通过相互合作,从对方获取自己没
有的数据块,从而有效地节约了数据源所在节点的网络资源。同时,这些工具能
够动态的适应网络环境的变化,选择最佳的数据传输路径,从而提高了部署的效
率,也能有效地应对一些网络故障。依据节点间形成的网络拓扑结构,可以将工
具分为基于随机结构网络拓扑的,或者基于结构化的网络拓扑的。

一些复杂的分布式应用通常由多种类型的节点构成。例如一个典型的三层结构的
Web应用,由前端的Web服务器,中间的应用处理和后端的数据存储构成。每一层
都可能是一个分布式系统。为了有效地支持这种复杂的部署需求,一些功能上更
先进的部署工具被研究并开发出来。这些工具提供了与自身绑定的描述语言,使
用者使用这些语言描述部署说明,使得工具能够自动完成指定的部署任务
(smartfrog, plush)。

\subsubsection*{监测}

分布式系统的运行状态散布在各个机器,人们需要监测系统的运行状态。这包括
底层计算平台的状态,与应用本身的状态。所监测的状态指标有表示系统是否正
确运行的,也有表示系统运行性能的。

在许多情况下,监测的目标并不是针对每个机器或者每个应用进程,而是针对所
有状态的一个聚集运算。例如,返回所有CPU使用率超过95\%的前十个机器列表。
最直接的办法是将所有状态集中收集到一起,然后进行计算。然而这样的方式扩
展性太差,在实际中没有可操作性。一些算法使用树的结构,将需要聚集的属性
从树叶向上传递,树中的节点将子节点的结果聚集后再向上传递,因此减少了数
据的传输。还有一些系统使用基于epidemic和gossip的算法,通过节点之间随机
的交换数据,达到统计意义上保证一定准确率的聚集结果。

使用聚集的方式可以持续监测某个全局状态的值,但是在一些情况下,人们只想
在知道全局状态是否满足一些约束谓词。例如,从每个机器访问任意站点的流量
总和不能超过100M/s,这是监测DDoS攻击的一个全局约束。在这种情况下,我们
不需要持续的聚集每台机器的流量状况。针对这种需求,可以使用分布式触发器
(distributed trigger)技术。这种技术在约束没有被违反的情况下并不传输
任何数据,因此进一步降低了检测所带来的流量负载。

\section{分布式系统调试与任务模型}

分布式系统是internet应用的关键组成部分。然而分布式系统很难设计与实现,
并且很容易包含错误(bug)。分布式系统将用户请求在很多机器上并发处理,
其目的是为了提高处理效率。然而并行程序不容易编写,需要考虑更多的运行状
态与处理逻辑。并行程序通常使用异步通信,以提高性能。因此消息可能是乱序
的,或者被丢弃。这使得处理逻辑更复杂。另外,在大规模系统中,机器与网络
故障是频繁发生的事件,分布式系统需要考虑到各种可能的情况并很好的处理。
考虑到以上这些困难,设计一个运行正确,并且可扩展性好、稳定、性能良好的
分布式系统是有相当挑战性的。

分布式系统的错误大致可以分为两类,正确性错误与性能错误。一个正确性错误
的例子是,在一个采用主从结构设计的分布式存储系统中,任何时刻只能有一个
主要节点。如果一些未考虑的事件序列,导致系统产生了两个主要节点,那么系
统就产生了正确性错误。性能错误的例子是,在一个搜索引擎服务中,前端服务
器接到用户搜索请求,将请求分发给后端的多个索引服务器,等所有的索引服务
器都返回后,将搜索结果返回给用户。然而,索引服务器的工作进度有快有慢,
因此整体的搜索性能受到最慢的索引服务器限制。在某些情况下,有的索引服务
器返回结果的时间很长,造成了性能问题。

分布式系统的错误难于调试,产生错误的原因深藏在系统复杂甚至是混乱的逻辑
与实现中。目前,设计分布式系统尚缺乏形式化方法支持,因此搭建分布式系统
更像是艺术,设计者凭借自己的经验进行设计。然而异步发生的消息传输与错误
事件,使得可能的系统状态爆炸性增长。因而总会有超出经验的错误发生。

目前,有以下一些分布式系统调试的方法。

\subsubsection*{基于任务模型的方法}

这种方法的基本思想是使用任务模型描述分布式系统的运行时行为,从而人们可
以使用任务模型理解、验证和分析分布式系统的设计与实现。

分布式系统将任务,例如一个用户请求,分为不同的子任务执行,这些子任务可
能被分散到系统的不同节点、进程和线程中处理,其执行顺序可能是并行也可能
是串行的。子任务之间,使用信号、异步消息等机制协调执行顺序。传统的调试
方法很难跟踪与分析分布式系统的运行时行为。

可以将分布式系统的运行时行为用任务模型来描述。任务模型描述了任务的执行
路径。任务模型可以用图来描述,包含两部分,组成任务执行过程的子任务,和
子任务之间的因果依赖关系。子任务是图上的节点。子任务之间边对应的它们之
间的依赖关系。子任务因为使用信号、异步消息等通信机制协调运行顺序,而产
生依赖关系。

使用任务模型描述系统运行时行为需要解决两个主要困难。首先,需要标记合理
的任务边界。分布式系统通常使用了线程复用机制(线程池)处理消息与事件。
这导致同一个线程在不同的时间段会执行不同的任务,线程与任务之间不存在一
一对应关系。因此,需要将线程的执行过程分为不同的片段,每个片段属于一个
任务的执行过程。其次,需要正确的联系任务之间的因果依赖关系。分布式系统
运行时并发产生许多的异步消息、信号、事件等通信,将它们扑捉并正确的与相
关联的任务配对并不容易。

\subsubsection*{基于谓词检查的方法}

可以使用谓词来检查分布式系统的不变量。虽然分布式系统非常复杂,但是其主
要性质可以用少数几个不变量来描述。例如,对一个主从结构的分布式存储系统
过来说,一些主要的不变量包括,数据副本一致性,任何时刻只能有一个主要节
点。只要这些不变量能够保持,系统就能够正确运行。

使用谓词检查分布式系统,首先需要动态提取计算谓词需要的系统状态,这些系
统状态分布在不同的节点和进程内。将状态汇总后,可以计算谓词是否满足。在
计算谓词时,应在系统的一组一致快照(consistent snapshot)上进行检查。

\subsubsection*{基于回放的方法}

分布式系统的错误通常依赖于一组特定的事件序列,错误不是确定性的。在错误
发生后,重新运行系统并不会产生相同的结果。基于回放的方法把系统运行过程
中的事件、消息与它们的顺序都记录下来,在错误发生后,能够重新“播放”错
误发生的过程,直到错误原因被找到为止。

有两种回放的方法。数据回放方法记录了每个进程运行时收到的消息,在回放时
原样提供这些数据。数据回放记录的数据量很大,因为需要记录每一条收到的消
息,但是可以在回放时分别调试每一个进程,不用回放整个系统。顺序回放的方
法不记录消息的内容,只记录消息发送的顺序,保证在回放时系统能够确定性的
重播错误发生的过程。顺序回放记录的数据小,但是回放时需要重新运行整个系
统,消息的内容在系统运行时动态产生。

% causality, execution path and task model
% 
% predicate
% 
% replay
% 
% model checker
% 
% 
% 分布式系统设计复杂,是为了满足可扩展性等要求。对调试有了新的困难。
% 
% 分析分布式系统的复杂点
% 
% 任务模型的作用
% 
% 因果关系图?
% 
% 日志(一般性研究的方法,结果)
% 
% 基于对分布式系统运行状态监测的分析?


\section{本文研究内容与贡献}

\subsection{研究内容:问题、挑战与基本方法}

\subsubsection*{可自管理的分布式应用管理系统}

为了有效地管理分布式应用,其管理系统也变得越来越复杂。管理系统本身也成
为了一个复杂的分布式应用,由散布在许多机器上的管理节点构成,共同完成应
用的部署、监测与控制等任务。

这就带来一个问题,也就是,由于管理系统变得越来越复杂,其本身也需要一定
的管理工作。需要部署管理系统的节点,并监测其是否正常运行。使用现有的技
术,有两种解决这个问题的方法。其一、使用集中式方法管理需要使用的管理系
统。集中式方法易于实现,其本身也易于被管理。但是集中式算法扩展性差,对
网络资源的利用效率低,同时不能够很好的应对分布式环境下经常发生的网络异
常。其二、使用一个采用P2P算法的管理系统去管理我们需要使用的管理系统。
这个方法虽然表面看上去很有效,但是新采用的管理系统又递归的产生了需要被
管理的问题,所以这个方案并没有解决问题。

在本文中,我们通过给管理系统增加内建的自管理能力,从根本上解决了管理系
统本身也需要被管理的问题。我们设计了一个具备自管理能力的分布管理系统
SMON(Self-Managed Overlay Network)。构成这个系统的节点相互监测运行状
态,并相互管理维护。SMON能自动将自己部署到一组指定的节点上去,也能够自
动恢复运行失败的节点,同时能够在线将整个系统更新至新的版本。

设计SMON有如下的几个难题。首先是保证其自管理具有良好的可扩展性,其次,
为了支持自动部署,需要一定的安全机制保证系统节点能够自动与计算平台的机
器认证并登陆,完成远程部署的任务。最后,在管理功能上,SMON应该具有良好
的可扩展性。

我们分别解决了这几个难题。首先,我们使用epidemic算法实现自管理。SMON节
点相互随机探测,并依据探测的结果,自动在远程机器上部署新的SMON节点,或
者恢复失败的SMON节点,或者更新自身至新的版本。其次,我们使用了认证代理
使SMON节点能够自动与计算平台的机器认证并登陆。SMON节点将登陆认证过程中
的挑战密文转发给认证代理,并返回认证代理求解的应答密文给远程机器,从而
完成自动认证的过程。同时,这一机制保证了用户提供的认证信息不会泄露到认
证代理外部。最后,由于SMON具有自管理功能,因此任何被SMON管理的分布式应
用也一并具有了自管理的能力。进而,可以在SMON上部署其它的分布式应用管理
系统,从而扩充了整个系统的管理能力。

我们在Planet-Lab平台上对SMON进行了评测,结果显示,SMON具有很好的扩展性
和性能。

\subsubsection*{自动推断系统层次结构任务模型的方法}

分布式系统难于调试。支撑internet服务的分布式系统本质上非常复杂,这增加
了理解系统异常行为的难度。这些系统通常使用了分层的体系结构,将其功能抽
象表达为不同的层次结构。其运行时具有高度的并发性。系统在运行时,处理着
许多用户层次的\emph{任务},例如用户请求。任务被分成许多阶段,不同的阶
段被分布在多个机器、进程和线程上执行,使用事件或者异步消息作为通知机制。
验证单独每个任务的行为,是一件具有挑战性的问题,因为开发人员需要重构出
任务的执行流,将任务执行过程中的各个阶段重新连接起来。

如图xxx所示,三层结构的Web服务系统结构。

从概念上说,开发人员可以把任务执行表示为\emph{层次结构的任务模型},这
与系统的分层体系结构设计一致。不同层次的任务表示在不同阶段不同模块执行
的一段执行过程。高层任务的执行被分为若干低层子任务。基于任务模型,开发
人员可以更好的理解系统模块间的结构,以及不同模块间的依赖关系,并验证处
于不同层次任务的行为。

已有工作需要开发人员手动标注任务模型。只有对系统的设计与实现非常熟悉的
专业人员才能正确的标注任务模型。针对这个问题,本文研究了一种自动推断系
统层析结构任务模型的方法。

这种方法使用插装技术,透明的监测系统运行,得到系统运行trace。在运行
trace基础上,我们自动推断出叶子任务的边界,叶子任务是任务模型中最基本
的任务单位。然后,我们推断出叶子任务之间的因果依赖关系,得到系统任务的
因果关系图。在这个因果关系图的上,我们使用图的聚类算法自动寻找频繁子图,
将叶子任务归纳为高层任务。通过递归使用聚类算法,我们得到了系统的层次结
构任务模型。

\subsubsection*{基于日志的系统任务模型推断方法}

我们进一步研究了如何利用系统日志,推断系统任务模型的方法。

与自动推断系统层次结构任务模型的方法不同,这个工作基于系统运行过程中生
成的日志。其理由是,系统日志包含了更多的应用层任务流的语意。日志是由系
统开发人员创建的,因此它们包含了关键的系统状态和事件信息,例如,用户层
任务的开始和结束,处理用户请求的关键步骤等等。利用这些信息,可以帮助我
们推断用户层语意,而这些很难从系统运行trace中得到。

使用日志推断分层任务模型需要解决两个问题。第一、日志通常是非结构化的,
我们需要从中提取出和任务相关的信息。第二、推断任务之间的层次关系,构建
层次结构的任务模型。我们通过观察系统日志得到了一个提取任务信息的启发,
使用这个启发,我们从日志中提取出了任务信息,包括任务的名称与ID。接着,
我们xxx

我们实现了基于日志的任务模型推断工具,并使用它分析一个分布式存储系统
ChunkFS。ChunkFS是一个类似GFS~\cite{gfs}的分布式文件系统。我们的工具能
够推断出合乎逻辑的任务模型。应用推断的模型,帮助我们理解ChunkFS的实际
运行过程,并解决了ChunkFS中的一个性能问题。经验表明,系统日志能够很好
的反映出上层语意,我们的工具可以有效的推断系统任务模型,帮助理解系统设
计和运行时行为。


\subsection{目标使用者}

本文针对分布式系统管理与调试的一些关键问题进行了一些研究,其研究结果可
以被以下三类使用者使用:

\begin{description}

  \item[系统开发者:] 系统开发者设计并实现系统,并负责调试与优化系统。他
  们对代码非常熟悉,知道系统运行时应该是什么行为。他们可以使用本文的工
  作验证系统设计是否正确,找到系统错误发生的根本原因。

  \item[系统代码维护者:] 系统代码维护者能够访问系统代码,但是对整个系
  统的设计并不熟悉。通常他们需要逐渐熟悉代码和系统设计。他们可以使用本
  文的工作理解系统的运行时行为,却不一定知道运行时行为是否正确。

  \item[系统管理员:] 系统管理员通常不需要接触系统代码。他们可以使用本
  文的工作有效地管理分布式系统。通常,系统管理员需要监测系统是否运行正
  确,他们也可以使用本文工作监测系统运行时行为是否和预期一致。系统的正
  确运行时行为由系统设计和开发者提供。

\end{description}

\subsection{研究贡献}

\subsection{内容组织}

