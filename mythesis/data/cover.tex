% vim:textwidth=70

%%% Local Variables:
%%% mode: latex
%%% TeX-master: t
%%% End:
\secretlevel{绝密} \secretyear{2100}

\ctitle{分布式系统管理与调试关键问题的研究}
% 根据自己的情况选,不用这样复杂
\makeatletter
\ifthu@bachelor\relax\else
  \ifthu@doctor
    \cdegree{工学博士}
  \else
    \ifthu@master
      \cdegree{工学硕士}
    \fi
  \fi
\fi
\makeatother


\cdepartment[计算机]{计算机科学与技术系}
\cmajor{计算机科学与技术}
\cauthor{高崇南} 
\csupervisor{郑纬民教授}
% 如果没有副指导老师或者联合指导老师,把下面两行相应的删除即可。
% 日期自动生成,如果你要自己写就改这个cdate
%\cdate{\CJKdigits{\the\year}年\CJKnumber{\the\month}月}
\cdate{\CJKdigits{2009}年\CJKnumber{4}月}

\etitle{Research on Key Problems of Distributed System Management and Debugging}

% \edegree{Doctor of Science} 
\edegree{Doctor of Engineering} 
\emajor{Computer Science and Technology} 
\eauthor{Chongnan Gao} 
\esupervisor{Professor Weimin Zheng} 
% 这个日期也会自动生成,你要改么?
\edate{April, 2009}

% 定义中英文摘要和关键字
\begin{cabstract}

% 800-1000汉字

  分布式系统成为了支撑Internet服务的关键组成部分,随着系统规模越来越大,
  系统处理逻辑日趋复杂,有效的管理与调试分布式系统成为了挑战。目前相关
  研究取得了一些进展,但很多关键问题尚待解决。本文针对分布式系统自管理
  机制、自动推断分布式系统层次结构任务模型的方法等方面进行了研究,取得
  了有价值的研究成果。

  本文主要贡献包括:

  \begin{enumerate}

    \item 分布式应用管理系统可自管理机制的研究。分布式系统需要被管理。分
    布式应用管理系统帮助人们简化了部署与维护分布式系统的负担。为了有效
    的完成管理任务,管理系统本身也是一个分布式应用,需要被管理,已有的
    工作还没有很好的解决这个问题。本文提出了使管理系统具备自管理机制的
    方法,并实现了一个可自管理的分布式应用管理系统SMON。SMON能够自动将
    自己安全部署到一组指定的机器上去,将自己从运行错误中恢复,并在线升
    级自己至新版本。

    \item 自动推断系统层次结构任务模型的方法。分布式系统难于调试,理解
    系统的运行时行为是分析解决系统正确性与性能问题关键。已有的方法可以
    帮助人们理解系统运行时的因果路径,但是需要程序员手工标注因果路径,
    使用困难且需要使用者对系统有深入了解。本文提出了一种完全不需要人工
    帮助,自动推断系统层次结构任务模型的方法。使用得到的层次结构任务模
    型,可以对系统设计有深入的了解,并帮助分析导致系统缺陷的原因。

    \item 利用系统日志推断系统任务层次结构的方法。日志是由系统开发人
    员创建的,因此它们包含了关键的系统状态和事件信息。利用这些信息,可
    以帮助推断应用层语意任务之间的关系。本文的推断方法能够自动从无结构的系
    统日志文本中提取任务信息,并推断任务之间的层次结构关系。使用得到的任务
    层次关系,可以帮助理解系统设计,解决系统已有的性能问题。

  \end{enumerate}

\end{cabstract}

\ckeywords{分布式系统; 自管理; 层次结构任务模型; 日志分析}

\begin{eabstract}
  Distributed systems are the key components in today's Internet
  services. While the system scale goes larger and the design
  logic becomes more complex, it is a challenge to managing and
  debugging large scale distributed systems. While current research
  work makes great progresses, some key problems are still not fully
  addressed yet. In this thesis, we conducted research on
  self-management of distributed systems and automatica inference
  of hierarchical task models of distributed systems.

  The contributions of this thesis are:

  \begin{enumerate}

    \item Mechanism of adding self-management property for distributed
    application management systems. Distributed systems need
    management. The distributed application management system
    simplifies tasks in deploying and maintaining distributed systems.
    To efficiently accomplish management tasks, management system is
    itself an distributed application, which needs to be managed too.
    Existing work doesn't address the problem very well. In this
    thesis, a mechanism to support self-management of distributed
    application management system is proposed and a self-managed
    distributed application management system (SMON) is designed. SMON
    can automatically and securely deploy itself to a set of machines,
    recover itself from failures and upgrade itself to
    newer versions online.

    \item Methodology of automatic inference of hierarchical tasks
    models for distributed system. It is a challenge to debug
    distributed systems and understanding the runtime behaviour is the
    key to solve the correctness and performance bug. Existing tools
    require manual annotation, which is hard to use and poses a high
    bar on the user. In this thesis, an automatic inference
    methodology of hierarchical task models for distributed systems is
    developed. The hierarchical task models help on understanding of
    system design and implementation, and help on finding root causes
    of system correctness and performance bugs.

    \item Methodology of inference of hierarchical task models based
    on system log. Logs are created by system developers and contain
    key information on system states and events. Using the
    information, we can understand the relations among application
    level tasks. The inference methodology can extract task
    information from unstructured system logs automatically and
    infer hierarchical relations among the tasks. The
    hierarchical task models help on understanding of system design
    and implementation, and help on finding root cause of system
    correctness and performance bugs.

  \end{enumerate}

\end{eabstract}

\ekeywords{Distribute System; Self-Management; Hierarchical Task
Model; Log Analysis}
