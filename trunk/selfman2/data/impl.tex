\section{Implementation}
\label{sec:impl}

We implement a prototype of SMON system on Planet-Lab.  The
source is written in Python, and the final installation
package size of a BON peer is about 122KB, with 1000+ lines
of code.

\begin{table*}
\small
\centering
\begin{tabular}{|l|l|}

\hline
\textbf{RPC} & \textbf{Description} \\

\hline
\texttt{ping(ver)} & Call with local SMON peer's version and
return remote peer's version.\\

\hline
\texttt{retrieve\_peer(ver)} & Retrieve installation package
of SMON peer with specified version.\\

\hline
\texttt{exchange\_livetag(tag, ver)} & Call with local
$<$livetag, version$>$, and return remote peer's $<$livetag,
version$>$.\\

\hline
\texttt{exchange\_member(ver)} & Call with local membership
list version and return remote peer's membership list.\\

\hline
\texttt{retreive\_member(ver)} & Retrive membership list of
specified version.\\

\hline
\texttt{exchange\_app(app\_name)} & Call with an application
name, return true if remote peer has installed the
application.\\

\hline
\texttt{retrieve\_app(app\_name)} & Retrieve installation package
of an application with specified name.\\

\hline
\texttt{resolve\_challenge(challenge)} & Return the response
to an authentication challenge.\\

\hline
\texttt{set\_app\_status(app\_name)} & Set application status (online,
offline, ignore).\\

\hline
\texttt{get\_app\_status()} & Get application status. \\

\hline

\end{tabular}
\caption{RPC interfaces of SMON peer and authentication
agent}
\label{fig:rpc}
\end{table*}

The epidemic activities of a SMON peer is implemented in
several
threads, including one for monitoring and maintaining other
peers, one for updating membership list, one for update
\texttt{livetag}. For each application, a separate thread
will be started for maintaining the application.

The communication with peers and agents are implemented as
RPCs, and they are summarized in table~\ref{fig:rpc}.


SMON peer copies installation packages or executes commands
by spawning a ssh/scp process. A modified \texttt{ssh-agent}
is started which forwards the authentication challenge from
ssh/scp to the remote authenticate agent. The agent only
implements one RPC interface \texttt{resolve\_challenge}
described in table~\ref{fig:rpc}.

The parameters used at SMON runtime is stored in a
configuration file. It stores the time intervals among
consecutive epidemic activities, the version numbers for
SMON peer and membership list, the $<$livetag, version$>$
tuple, the address of authentication agent. Each application
can specify an address to which the application's states
changes should be reported. The configuration file is
implemented as a SQLite database file to avoid data loss at
machine crash. It is updated by
installation package of SMON and applications.  The
membership list is stored separately in compressed format.

%SMON makes a predefined and consistent directory layout of
%peer installation and it is critical for peers to perform
%their functions correctly. A SMON peer is installed to
%\texttt{SMON} directory of user's home directory, and this
%setting can be changed. Under \texttt{bin} directory, a
%\texttt{run.py} and \texttt{smon.cfg} is placed. To start
%SMON peer, \texttt{run.py} is executed and it read
%configuration in \texttt{smon.cfg} and start the version
%specified in \texttt{smon.cfg}. A new version SMON peer
%should change \texttt{smon.cfg} appropriately. Different
%version of SMON peer is placed according to version number
%under different directories. \texttt{app} directory contains
%different application entries, as different directories. A
%configuration file for each application is placed under it
%directory. Installation package of SMON peer and
%applications are stored under \texttt{packages} directory,
%placed according to its name and version.

%ping interval and timeout value

% vim:foldmethod=marker:textwidth=60
