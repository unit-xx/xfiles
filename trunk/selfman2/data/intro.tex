\section{Introduction}

% flow没太大问题。

Distributed systems~\cite{Ghemawat2003, DeCandia2007} are at
the heart of today's Internet services. While the cost of
commodity computers continues to drop, it is common for
distributed computing platforms to contain hundreds or
thousands of computers, such as
Planet-Lab~\cite{Bavier2004}, Amazon
EC2~\cite{Garfinkel2007} and Teragrid~\cite{Catlett2002}.
These platforms can provide huge amount of storage and
computing capabilities that boost performance of distributed
applications significantly. However, it is still difficult
to design, deploy and manage distriubted applications at
large scale and fully utilize the resources.
%handle frequent and inevitable failures at host,
%application and network levels.

%As the scale of distributed applications become larger than
%ever, it is hard to manage them efficiently on distributed
%platforms.  Managing a distributed application involves
%several tasks.  The application must be first deployed to a
%set of computers. After the application is configured and
%started, its status need to be monitored for detecting and
%recovering from failures, ensuring sustained performance or
%just for collecting runtime data and statistics.
%Administrators may also want to send commands to control
%the application on demand.

% describe as a scenario: management system is needed in
% this way
Managing a distributed application faces several challenges.
The application must be first deployed to a set of machines.
The deployment process should have good scalability so that
it can work on thousands of machines. During the deployment,
failures---such as machine or network failures---are
inevitable. They must be handled at any time during file
transfer and application installation. Only when the
application is deployed and configured successfully can the
execution begin. After the application is started, certain
mechanisms are need to monitor and recover application from
failures.  Depending on application's semantic, the recovery
may be simply restarting the failed processes. For some
cases, failures may lead to aborting and restarting the
whole application on many machines. For all the
applications, quick failure detection and recovery is
required to ensure that applications are running
successfully. Distributed applications are also commonly
upgraded for bug fixes and performance improvement.

%After the application is configured and started, its status
%need to be monitored for detecting and recovering from
%failures, ensuring sustained performance or just for
%collecting runtime data and statistics.  Administrators may
%also want to send commands to control the application on
%demand.


% Distributed application management system is designed to
% simplify tasks involved in deploying and maintaining the
% applications after they are designed and implemented.  The
% management system needs to deploy a distributed
% application to a set of machines and start it. After that,
% it monitors the application and 

% Distributed application management system is designed to
% simplify tasks involved in deploying and maintaining the
% applications after they are designed and implemented. They
% hide the underlying details and provide a user-friendly
% interface for developers to manage applications on many
% machines automatically. The developers can focus on
% bug-fixing and performance tunning of their applications,
% rather than deploying, monitoring and upgrading the
% applications on a set of distributed machines.

% automation: the importance of management system
Distributed application management system is designed to
ease the burdens of managing applications on many machines.
It automates many aspect in deploying and maintaining
applications. By hiding the underlying details and providing
a user-friendly interface, distributed applications can be
deployed, monitored, recovered and upgraded automatically
given the instructions from developers. The developers can
focus on bug-fixing and performance tunning of their
applications, rather than managing the applications on a set
of distributed resources.

% ok here, just ignore the simple tools like vxargs. Don't
% regard them as `system' the root cause: they are
% distributed applications in fact.
To manage distributed applications at large scale, the
management systems have to take distributed designs, or to
say, they are also distributed applications. A mangement
system either adopts centralized (or client/server-like) or
peer-to-peer (P2P) approach.  Generally, a distributed
management system consists of many ``peers''\footnote{The
word peer here will refer to client if the management system
is in client/sever architecture.} on each machine where
application will be deployed. The peers form an overlay
network, and they work corporately to deploy application to
a set of machines. After the application is started, each
peer monitors and maintains application's processes within
the same machine box.
%User can query the applications status
%and control them on demand with the help of management
%system. XXX: need to say this?
%User can send messages into the overlay to query
%application status and execute control commands on demand.

% the caveat: XXX we still need to prove it is really a big
% problem without selfman: we have idea to manage app, but
% we cannot do as better to manage manager.  have to be
% vivid: how about an example of bt? appman?  plush?  how
% about this: bt as example -> plush do bootstrap in
% centralized way -> not efficient and very preliminary
There is one important caveat regarding the use of an
application management system: the management system itself
should be deployed first and maintained continually
throughout its lifecycle, which is not easier than managing
any other distributed applications. Intuitively, we can use
an existing management system $M'$ to depoy and maintain
another management tools $M$. But this approach doesn't
\emph{solve} but just \emph{migrate} the problem.
Currently, developers resort to writing scripts---which
leverage OS services such as sshd on Unix/Linux---to
maintain management systems centrally. This approach is very
preliminary and has several disadvantages. First...

% Developers resort to
% xxx (simple script leveraging OS services like sshd to
% deploy, recover and upgrade the management system).
% More complex and advanced tools deosn't help xxx. (倒腾
% 清楚目前的办法,和他们的不足) Using other tools or services
% don't solve but just mitigate the problem. Because they all
% have clients/peers to take care of.
% 
% what we want? combine the virtues (maybe not the virtues). And then we have
% self-man. what's selfman, how it works (the general idea),
% the key challenges.
% 
% 
% 1. the management system must be managed, this is ok, any
% distributed application requires that
% 
% 2. management of the management system is not well
% supported: a) using another p2p tool? b) using another
% centralized tool?
% 
% 3. the management system is critical, if it has problem
% because of itself's management, all the applications will be
% affected. Any example?
% 3.5 find a scenario to motivate self-man
% 
% 4. the self-man problem is common for ALL management
% systems.
% 
% 5. prevent it from automating every steps in app management
% 6. not scalable using centralized
% 7. management tools need bootstrap, while they bootstrap
% other applications.
% 
% 现状:1. 用codeploy,sharkfs
% 2. 用vxargs
% 3. 用plush,cfengine
% 
% 无论用别人的service还是tool,都有人需要维护管理系统,这个工
% 作职能用centralized的方法做,不管是使用外部工具,还是管理工
% 具本身集成了一些自管理的能力。
% 
% 问题:使用已有的稳定的service管理其他service?
% 
% 问题:在计算平台上,已有一套管理系统,使用者不用费力,管理者
% 也只有一套系统需要维护,并没多大问题。

% The distributed design makes management system a distributed
% application in essential and it introduces an important
% problem: the management system need to be deployed first and
% maintained continually through its lifecycle. Currently,
% this problem is addressed by using centralized approach. A
% central controller first deploys and starts the peers of
% management system on a set of machines. After that, the
% controller periodically monitors the peers and recovers the
% failed ones. If a crashed machine is replaced by a new one,
% the controller has to deploy a new instance of peer on the
% fresh machine. The controller can also upgrade peers to new
% versions with bug fixes and improved performance. The
% centralized approach has several disadvantages. First, it is
% not scalable at deploying peers. Second, it is not efficient
% at monitoring the peers. Frequent monitoring detects peers
% failures quickly, but it puts heavy burden on network
% resource of the central controller when establishing
% connections and sending messages to remote machines. Third,
% The static star topology from central controller to peers
% cannot handle changes or failures in network
% environments, such as network partition.
%What is more, trying to use more advanced management
%systems which use their own peers to manage another one
%will introduce the same problem recursively.  \note{define
%managment loop here, and we automate the loop within peers
%in SMON}

In this paper, we propose Self-Managed Overlay Network
(SMON) that addresses the problem of deployment and
maintenance of distributed management systems.  SMON is a
distributed management system with built-in self-management
capability, namely, self-deployment, self-recovery and
self-upgrade. It consists of peers on every target machines
where applications will be deployed and maintained. The
peers of SMON monitor and maintain other ones running on
different machines. A SMON peer will automatically deploy
new peers on fresh machines, recover failed peers, or
upgrade peers of old versions. The collective behavior of
all peers gives rise to a distributed system that can deploy
, recover  and upgrade itself automatically. SMON can also
manage a set of applications.
%especially another distributed application management
%system. In this way, SMON's management functionality can
%extended greatly.

% XXX: the advantage of selfman are:

\comment{
Running SMON peers will automatically deploy
and start new instances of SMON peers on machines where SMON
peers are not installed.  In this way, operator only needs
to deploy and start one SMON peer and SMON will be deployed
to all the target machines quickly.

Each SMON peer has an associated version number and it is
stored persistently in configuration file. The version
number can be used to update SMON ``online'' to newer
versions.  When two peers communicate with each other and
find a difference in version, they work corporately to
update the lower version to the latest version. Using
epidemic communication, the whole SMON system will be
updated eventually once a single peer is updated.

SMON may be stopped because of machine failures or other
reasons. The failed peers will be detected and started
by running ones automatically.
}


%SMON can deploy itself to a set of target machines
%automatically. While it is running, it monitors itself and
%recovers failed peers. If a new version of SMON is
%available, it will update itself online. SMON can also
%deploy and maintain a set of distributed applications. User
%can use a set of management interfaces to query status and
%set parameters of SMON and managed applications.

In designing SMON there are several challenges.

The first challenge is that SMON should have good
scalability.  Second, there should be built-in security
mechanism so that peers can automatically login into other
machines to deploy new peers or recover failed peers. And
access to SMON should be authenticated and encrypted to
avoid misuse of the system for malicious activities.  The
Third challenge is that SMON should define reasonable
application management semantic.

SMON's design addresses the challenges as follows. For the
first challenge, SMON peers monitor and maintain each other
using epidemic approach. This ensures good scalability
($O(\log N)$) when system goes large. For the second
challenge, a separate authentication agent is used to store
and protect login credentials. When a peer needs to login
into another machine, it redirects the authentication
challenge from remote machine to the agent and replies the
response solved by the agent to the machine.  The credential
is never leaked out of the agent. The authentication load on
the agent is very light and a single agent can support a
number of peers. We use a symmetric key to
authenticate and encrypt access to SMON peers.
For the third challenge, SMON defines xxx. To extend xxx.

%It is of great challenge to manage distributed application
%when the scale is large. Distributed application management
%system is designed to ease the burdens of deploying and
%maintaining distributed applications in large-scale
%computing platforms. It provides user-friendly interfaces for
%people to deploy, configure, monitor and control distributed
%applications.

%To achieve
%these goals, a management system faces several challenges.
%It must have good scalability so as to deploy applications
%and broadcast control commands efficiently.  Host or network
%failures must be carefully handled to reduce their negative
%effects on application management to near minimum. When
%application is started, the management system monitors its
%running status. It must provide a extensible way for
%developers to define abnormal status of the application and
%the corresponding actions to peformance against abnormal
%status.

In summary, the paper makes following contributions.  We
design a scalable, secure and extensible distributed
management system with built-in self-management capability
based on epidemic algorithm.  We implement it on Planet-Lab
platform.  The evaluation shows that SMON has good
performance and achieves good scalability.

The rest of paper is organized as follows. We describe SMON
design in section~\ref{sec:design}. Section~\ref{sec:impl}
describes our implementation on Planet-Lab platform. We
present evaluation of SMON in section~\ref{sec:eval}.
Section~\ref{sec:related} relates SMON with previous systems
and section~\ref{sec:conclusion} concludes.

\comment{
\note{what is involved in management activity}

\note{the problem in current management tools:}

\note{our solution}

\note{the challenges}

scalable, robust and simple (simple made robust possible)

and extensible?

\note{how we address these challenges}

\note{a short summary on BON status/results}

\note{contribution}

\note{paper layout}
}
% vim:foldmethod=marker:textwidth=60

