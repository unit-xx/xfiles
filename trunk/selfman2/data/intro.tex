\section{Introduction}

% flow没太大问题。

Distributed systems~\cite{Ghemawat2003, DeCandia2007} are at
the heart of today's Internet services. While the cost of
commodity computers continues to drop, it is common for
distributed computing platforms to contain hundreds or
thousands of computers, such as
Planet-Lab~\cite{Bavier2004}, Amazon
EC2~\cite{Garfinkel2007} and Teragrid~\cite{Catlett2002}.
These platforms can provide huge amount of storage and
computing capabilities that boost performance of distributed
applications significantly. However, it is still difficult to
design and manage distriubted applications at large scale
and fully utilize the resources.
%handle frequent and inevitable failures at host,
%application and network levels.

%As the scale of distributed applications become larger than
%ever, it is hard to manage them efficiently on distributed
%platforms.  Managing a distributed application involves
%several tasks.  The application must be first deployed to a
%set of computers. After the application is configured and
%started, its status need to be monitored for detecting and
%recovering from failures, ensuring sustained performance or
%just for collecting runtime data and statistics.
%Administrators may also want to send commands to control
%the application on demand.

Managing a distributed application faces several challenges.
The application must be first deployed to a set of machines.
The deployment process should scale well so that it still
works on thousands of machines. During deployment, 
%After the application is configured and started, its status
%need to be monitored for detecting and recovering from
%failures, ensuring sustained performance or just for
%collecting runtime data and statistics.  Administrators may
%also want to send commands to control the application on
%demand.


% Distributed application management system is designed to
% simplify tasks involved in deploying and maintaining the
% applications after they are designed and implemented.  The
% management system needs to deploy a distributed
% application to a set of machines and start it. After that,
% it monitors the application and 

Distributed application management system is designed to
simplify tasks involved in deploying and maintaining the
applications after they are designed and implemented.
To
efficiently manage distributed applications at large scale,
the management system is designed in distributed approach.
It consists of many ``peers''\footnote{The word peer here
will refer to client if the management system is in
client/sever architecture.} on each machine where
application will be deployed. The peers form an overlay
network, and they work corporately to deploy application to
a set of machines.  After the application is started, each
peer monitors and maintains application's processes within
the same machine box. User can query the applications status
and control them on demand with the help of management
system.
%User can send messages into the overlay to query
%application status and execute control commands on demand.

The distributed design makes management system a distributed
application in essential and it introduces an important
problem: the management system need to be deployed first and
maintained continually through its lifecycle. Currently,
this problem is addressed by using centralized approach. A
central controller first deploys and starts the peers of
management system on a set of machines. After that, the
controller periodically monitors the peers and recovers the
failed ones. If a crashed machine is replaced by a new one,
the controller has to deploy a new instance of peer on the
fresh machine. The controller can also upgrade peers to new
versions with bug fixes and improved performance. The
centralized approach has several disadvantages. First, it is
not scalable at deploying peers. Second, it is not efficient
at monitoring the peers. Frequent monitoring detects peers
failures quickly, but it puts heavy burden on network
resource of the central controller when establishing
connections and sending messages to remote machines. Third,
The static star topology from central controller to peers
cannot handle changes or failures in network
environments, such as network partition.
%What is more, trying to use more advanced management
%systems which use their own peers to manage another one
%will introduce the same problem recursively.  \note{define
%managment loop here, and we automate the loop within peers
%in SMON}

In this paper, we propose Self-Managed Overlay Network
(SMON) that addresses the problem of deployment and
maintenance of distributed management systems. SMON is a
distributed management system with built-in self-management
capability, namely, self-deployment, self-recovery and
self-upgrade. It consists of peers on every target machines
where applications will be deployed and maintained. The
peers monitor each other and automatically deploy new peers
on fresh machines, recover failed peers, or upgrade peers of
old versions. The collective behavior of all peers gives
rise to a distributed system that can deploy itself to a set
of machines, recover failed peers and update itself to new
versions. SMON can also manage a set of applications.
%especially another distributed application management
%system. In this way, SMON's management functionality can
%extended greatly.

\comment{
Running SMON peers will automatically deploy
and start new instances of SMON peers on machines where SMON
peers are not installed.  In this way, operator only needs
to deploy and start one SMON peer and SMON will be deployed
to all the target machines quickly.

Each SMON peer has an associated version number and it is
stored persistently in configuration file. The version
number can be used to update SMON ``online'' to newer
versions.  When two peers communicate with each other and
find a difference in version, they work corporately to
update the lower version to the latest version. Using
epidemic communication, the whole SMON system will be
updated eventually once a single peer is updated.

SMON may be stopped because of machine failures or other
reasons. The failed peers will be detected and started
by running ones automatically.
}


%SMON can deploy itself to a set of target machines
%automatically. While it is running, it monitors itself and
%recovers failed peers. If a new version of SMON is
%available, it will update itself online. SMON can also
%deploy and maintain a set of distributed applications. User
%can use a set of management interfaces to query status and
%set parameters of SMON and managed applications.

In designing SMON there are several challenges.

The first challenge is that SMON should have good
scalability at monitoring and maintaining itself.  Second,
there should be built-in security mechanism so that peers
can automatically login into other machines to deploy new
peers or recover failed peers. And access to SMON should be
authenticated and encrypted to avoid misuse of the system
for malicious activities.  The Third challenge is that SMON
should have good extensibility.

SMON's design addresses the challenges as follows. For the
first challenge, SMON peers monitor and maintain each other
using epidemic approach. This ensures good scalability
($O(\log N)$) when system goes large. For the second
challenge, a separate authentication agent is used to store
and protect login credentials. When a peer needs to login
into another machine, it redirects the authentication
challenge from remote machine to the agent and replies the
response solved by the agent to the machine.  The credential
is never leaked out of the agent. The authentication load on
the agent is very light and a single agent can support a
number of peers. We use a symmetric key to
authenticate and encrypt access to SMON peers.
For the third challenge, SMON can be easily extended by
upgrade itself to a new version with new features or
improved performance. It can also manage other distribute
management systems so the management functionalities can be
extended greatly.

%It is of great challenge to manage distributed application
%when the scale is large. Distributed application management
%system is designed to ease the burdens of deploying and
%maintaining distributed applications in large-scale
%computing platforms. It provides user-friendly interfaces for
%people to deploy, configure, monitor and control distributed
%applications.

%To achieve
%these goals, a management system faces several challenges.
%It must have good scalability so as to deploy applications
%and broadcast control commands efficiently.  Host or network
%failures must be carefully handled to reduce their negative
%effects on application management to near minimum. When
%application is started, the management system monitors its
%running status. It must provide a extensible way for
%developers to define abnormal status of the application and
%the corresponding actions to peformance against abnormal
%status.

In summary, the paper makes following contributions.  We
design a scalable, secure and extensible distributed
management system with built-in self-management capability
based on epidemic algorithm.  We implement it on Planet-Lab
platform.  The evaluation shows that SMON has good
performance and achieves good scalability.

The rest of paper is organized as follows. We describe SMON
design in section~\ref{sec:design}. Section~\ref{sec:impl}
describes our implementation on Planet-Lab platform. We
present evaluation of SMON in section~\ref{sec:eval}.
Section~\ref{sec:related} relates SMON with previous systems
and section~\ref{sec:conclusion} concludes.

\comment{
\note{what is involved in management activity}

\note{the problem in current management tools:}

\note{our solution}

\note{the challenges}

scalable, robust and simple (simple made robust possible)

and extensible?

\note{how we address these challenges}

\note{a short summary on BON status/results}

\note{contribution}

\note{paper layout}
}
% vim:foldmethod=marker:textwidth=60

