\section{Related Work}
\label{sec:related}

% vxargs, appmanager, opt: smartfrog, cfengine
Many systems have been proposed to help managing distributed
applications.
%vxargs~\cite{vxargs} is a Python script which can run any
%command in parallel, including ssh, scp, etc. An important
%reason to use it is to control a large set of machines in
%the wide-area network. It consists of only a single script
%and easy to use, but has limited management support. 
PlanetLab Application Manager (appmanager~\cite{appmanager})
is designed mainly for managing long running services and
utilizes a simple client-server model. The clients actively
report recent status to and receive new instructions from
server.  Plush~\cite{Albrecht2007} is an extensible
management system in client-server model for large-scale
distributed systems, including Planet-Lab and Grid. With
user provided management specification, it handles resource
discovery (using SWORD), deployment, maintenance and
execution flow automatically. It has built-in barrier
support and can manage grid-style applications which run in
sychronized phases.  Plush-M~\cite{Topilski2008} further
improves Plush's scalability by replacing star topology with
RanSub~\cite{Kostic2003}.  SmartFrog~\cite{smartfrog} is a
framework for building and deploying distributed
applications. SmartFrog daemons running on each
participating node work together to manage distributed
applications. Unlike Plush, there is no central point of
control in SmartFrog: work-flows are fully distributed and
decentralized.  Although many management systems have been
proposed, their self-management problem is not fully
addressed.  User has to set up and maintain a management on
their own, usually using centralized approach. Plush can
bootstrap (self-deploy) its clients automatically, but the
centralized approach limits its scalability and makes no
difference.  SMON provides a scalable and secure
self-managment mechanism using epidemic algorithm. It turns
any managment system ``self-managable'' by deploying it upon
SMON.

% + Automatic computing
Researchers have been studying self-* properties of
distributed systems for long time.
Self-stabilization~\cite{Dijkstra1974} is a concept of
fault-tolerance in distributed computing. A
self-stabilization system will converge to a legitimate
state from any state. DHTs~\cite{Stoica2001, Ratnasamy2001,
Rowstron2001} are distributed systems that have
self-organizing, self-healing and self-tuning properties.
They will automatically organize their overlay topologies
conforming to predefined constraints, and handle network
churns automatically. \cite{Yin2008} proposes a self-hosting
system that acquires or releases resources dynamically and
automatically deploys the system to acquired resources (as
real or virtual machines) using underling system management
tools. SMON is complementary to these systems and these
techniques can be combined together to build distributed
systems which are more stable and efficient with little
human management efforts.

\comment{
DHT (self-organization),

Dependable Self-Hosting Distributed Systems Using Constraints
self-stabalization

19. E.W. Dijkstra, "Self-Stabilizing Systems in Spite of Distributed Control," Comm.
ACM, vol. 17, no. 11, 1974, pp. 643-644.

20. S. Dolev, Self-Stabilization, MIT Press, 2000.

plush LASCO provides improved scalability and fault
tolerance
}



% vim:foldmethod=marker:textwidth=60
