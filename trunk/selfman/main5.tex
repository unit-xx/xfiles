\documentclass[10pt, twocolumn]{article}

% preamble {{{
\usepackage[T1]{fontenc}
\usepackage{url}
\usepackage{natbib}
%\usepackage{lucidabr}
%\usepackage{times}
\usepackage[height=8in, margin=1in]{geometry}
\usepackage{indentfirst}
\usepackage{graphicx}

\bibpunct{[}{]}{,}{a}{,}{,}

%\title{\bf \huge BON: Bootstrap-enabled Overlay Network}
\title{\bf \huge Towards An Self-managed Tool For Distributed
Application Management}
\iffalse
\author{Chongnan Gao, Hongliang Yu, Jing Sun, Weimin Zheng}
\date{Tsinghua University}
\fi

\begin{document}

\maketitle
% }}}

% Abstract {{{
\section*{Abstract}

% }}}

% Introduction {{{
\section{Introduction}

Managing distributed applications on distributed computation
platforms \citep{Peterson2002, Foster2001} and large scale data
centers is a time-consuming and error-prone process.  It
involves deploying, monitoring and control of applications on a
set of distributed resources.  After the initial deployment of
an application, it must be monitored continually for detecting
and recovering from inevitable machine and network failures. For
availability and reliability considerations, the application
must be carefully monitored and controlled to provide continued
operation and sustained performance under dynamic fluctuations
in distributed environments. Currently, A number of tools have
been proposed and developed to address various aspects of
management process in distributed environments: resource
monitoring and discovering, content distribution, application
monitoring and control, etc.

\iffalse
They can
be categorized into two approaches: centralized approach and
peer-to-peer approach. While the centralized approach is easier
to design, develop and use, they cannot scale well for large
systems. The peer-to-peer approach scales better than the
centralized one.  And it can adapt to fluctuations in
distributed environment dynamically, if carefully designed.
\fi

The tools for managing distributed applications are themselves a
kind of distributed applications with `management' as their
essential functionality. So they also need to be well managed to
ensure that management activities conducted by users are
effective and efficient executed.

Take using BitTorrent to deploy content or applications as an
example. To use BitTorrent to deploy contents, the clients of
BitTorrent have to be deployed on each target node first.  After
starting all the BitTorrent clients, the running status and
deployment progress of each BitTorrent client have to be
monitored also.  When the deployment is completed, the
BitTorrent clients need to be stopped by user.  The situation
will be more complex when some clients cannot be monitored or
controlled properly because of machine and network failures.
Although BitTorrent is quite good at deploying content and
applications, much extra management effort is needed to use it.

\iffalse
The situation is similar for the
centralized management tools. The central controller and every
clients have to be deployed and managed.  Some centralized tools
make use of services provided by operating systems as
clients---e.g. sshd of *nix systems---and avoid the deployment
and management of these services but the management functions
are limited.  It is a choice to use management services that are
already deployed and maintained by others, which kicks the ball
back to the operators of the services. 
\fi

The extra effort needed for mananging distributed application
managent tools makes life painful. It also explains why some
simple tools in centralized approach, like vxargs, are popular.
Vxargs is a script tool which leverages underlying operating
system services (sshd on Linux) to deploy content, sending
control commands and querying applications status from a central
point. It is easy to use and exposes no extra management effort.
However, its functions are limited by Linux shell, and it is not
efficient at handling large volumns of data or large number of
nodes. When network partition occurs, the centralized
communication approach may cause large portions of nodes out of
touch.

This gives rise to the motivation of the paper. In this paper,
we propose a novel distributed application management tool xxx which
combines the virtues of BitTorrent-like and vxargs-like tools by
building self-managment capability as its foundamental
characteristic.  With self-management, it automatically deploys
itself to the computing platform. While it's running, it
actively monitors itself and recovers from machine failures as
soon as possible.  It leverages peer-to-peer approach and uses
epidemic algorithms extensively to conducte management
activities efficiently, and also to get around of network
failures. It is not designed as a full-featured management tool
with rich management semantic support. However, it implements a
basic best-effort semantic and a minimum set of RPC interfaces
for management, which is enough for managing common distributed
applications. To gain more management features, other
distributed application management tools, such as Plush, can be
deployed and managed by xxx. In this way, user gains the
required managment features while eliminating the tedious extra
management effort.

\iffalse
The challenges:

1. moderate authentication and encryption mechanism is need to
protecting user credentials. It is used at `login' step of
self deployment. The mechanism is also needed to prevent
un-authorized users to use xxx.

2. 
\fi

%}}}

recover of daemon from failure


\bibliographystyle{ieee} \bibliography{ref}

\end{document}

% vim:foldmethod=marker:textwidth=64
