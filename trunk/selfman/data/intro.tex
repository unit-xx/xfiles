\section{Introduction}

\note{management is a critical problem}

Distrbuted systems are at the heart of today's Internet
services (\note{cloud?}). While the cost of commodity
computers continues to drop, it is common for large scale
clusters to contain thousands of computers, such as
PlanetLab, Amazon EC2 and Teragrid. These platforms can
provide petabytes of data storage and hundreds of teraflops
of computing capabilities that boost performance of
distributed applications significantly. However, the
applications must be carefully designed to scale to
thousands of computers and fully utilize the resources.
%handle frequent and inevitable failures at host,
%application and network levels.

After distributed application is designed and implemented,
an important problem is how to manage it effectively and
efficiently on distributed platforms. Managing a distributed
application involves several tasks. The application must be
first deployed to a set of computers. Appropriate control
commands must be disseminated to configure and start
application processes. After the application is started, its
status need to be monitored for detecting and recovering
from failures, ensuring sustained performance or just for
collecting statistics. Operators may also want to send
control commands on demand to recover failed processes and
fine tunning application performance.

To efficiently manage distributed applications at large
scale, the manangement system is designed in distributed
approaches. It consists of many ``peers'' on each machine
where distributed application will be deployed. The peers
work cooperately to disseminate application's software
package and data. After the applicaiton is started, a peer
monitors local applicaiton processes within the same machine
box.

The distributed design makes management system a distributed
application in essential and it introduces a problem: the
management system need to be deployed first and maintained
continually through its lifecycle. To address this problem,
operator can use simple scripts which leverage OS's local
service (e.g. sshd) to ``manage'' a management system
centrally. It is noticed that more advanced tool will
introduce the same problem recursively.

%It is of great challenge to manage distributed application
%when the scale is large. Distributed application management
%system is designed to ease the burdens of deploying and
%maintaining distributed applications in large-scale
%computing platforms. It provides user-friendly interfaces for
%people to deploy, configure, monitor and control distributed
%applications.

%To achieve
%these goals, a management system faces several challenges.
%It must have good scalability so as to deploy applications
%and broadcast control commands efficiently.  Host or network
%failures must be carefully handled to reduce their negative
%effects on application management to near minimum. When
%application is started, the management system monitors its
%running status. It must provide a extensible way for
%developers to define abnormal status of the application and
%the corresponding actions to peformance against abnormal
%status.



management system need to be managed too. current approach
is not scalable. 
Distributed application management sysmtems help... To
manage distributed systems, the management system itself is
a distributed system in essential, and it has the same
problem on management with other distributed systems. It has
to be deployed... As far as we know, these challenges have
not been address in literature yet.

Distributed managment systems cannot rely on external
systems to solve the problems. and we proposed a
self-management scheme.


\note{what is involved in management activity}

\note{the problem in current management tools:}

\note{our solution}

\note{the challenges}

\note{how we address these challenges}

\note{a short summary on BON status/results}

\note{contribution}

\note{paper layout}

% vim:foldmethod=marker:textwidth=60

