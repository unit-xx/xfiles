\section{Implementation}
\label{sec:impl}

We implement a prototype of SMON system on Planet-Lab.  The
source is written in Python, and the final installation
package size of a BON peer is about 122KB, with 1000+ lines
of code.

SMON makes a predefined and consistent directory layout of
peer installation and it is critical for peers to perform
their functions correctly. A SMON peer is installed to
\texttt{SMON} directory of user's home directory, and this
setting can be changed. Under \texttt{bin} directory, a
\texttt{run.py} and \texttt{smon.cfg} is placed. To start
SMON peer, \texttt{run.py} is executed and it read
configuration in \texttt{smon.cfg} and start the version
specified in \texttt{smon.cfg}. A new version SMON peer
should change \texttt{smon.cfg} appropriately. Different
version of SMON peer is placed according to version number
under different directories. \texttt{app} directory contains
different application entries, as different directories. A
configuration file for each application is placed under it
directory. Installation package of SMON peer and
applications are stored under \texttt{packages} directory,
placed according to its name and version.

%ping interval and timeout value

% vim:foldmethod=marker:textwidth=60
