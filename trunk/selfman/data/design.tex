\section{Design}

Generally, SMON is defined as a distributed system with
built-in self-management capability, namely, self-deploy,
self-recover and self-update. It consists of peers on every
machine where applications will be deployed and managed by
SMON. The peers are all identical in functionality. They
communicate with each other epidemically to detect and
recover failed peers. A failed peer is not deployed or
killed ... So, for a failed peer, deployed first, if
necessary, and be started. In this way, even there is only
one peer, the whole SMON will be deployed and running on all
the machines ... While SMON is running, the failed process
will... SMON peer has an associated version number and it is
stored persistently with each peer. The version number is
used to update SMON ``online'' to newer version. When two
peer communicate ...  Two peers work cooperately to update
the lower version to the latest verison. Using epidemic
communication, this mechanism will ensure the whole SMON
system will be updated once a single peer is updated.

SMON can manage a set of distributed applications.

SMON can be used in several ways.

We then describe SMON design in details.

\subsection{Self-deployment}

For a list of target machines, SMON can deploy itself to all
the machines automatically. At the very beginning, there is
not any SMON peer deployed and running yet. At least one
peer must be deployed to a node and started manually.  The
first peer then detects other machines in its partial view
and deploy new peers. The new peers repeat the similar
detect-and-deploy process. While there are more SMON peers
deployed and running, this self-deployment process speeds up
exponentially. It is proved that with probability 1 all the
reachable machines can be deployed
eventually\cite{Eugster2004}.

The deployment precedure for a SMON peer works as follows.
A peer $P$ choose a random machine $M$ from its member list
and sends a ping message. If a pong message is not received
within predifined timeout intervals, it will start the
deployment precedure on machine $M$. $P$ first authenticates
itself with $M$ and login into $M$ automatically (described
in in \ref{subsec:security}), then it copies an installation
package of SMON peer to $M$, starts the packag and logout
$M$. The installation package will check the interity of
itself (currently using MD5 checksum) in case of transfer
errors during copy process. And it will configure and start
the SMON peer.

There may be failures (e.g. connection corrupted, machine
downed) at any time when peer $P$ authenticates and login
into $M$, copies and starts installation package remotely.
When failure happens, $P$ will just abort the deployment
precedure. $M$ will be choosen by another SMON peer at later
time. It is expected that a new SMON peer will be deployed
on $M$ eventually.

Because peers communicate with each other epidemically, it
is possible that two or more peers try to deploy SMON peer
on the same machine simultaneously. This problem is solved
without direct coordination among peers. Before deployment
procedure, a peer will generate a temporary directory name
using name of the machine in which it is running. The
installation package will be copied to that directory of the
remote machine, and it will not overwrite packages from
other peers. The execution of simultaneously started
installation packages are serialized using OS provided
sychnization facilities (currentl lock file). So only the
first installation process will finish and others will abort
because they find that a SMON peer is already running. In
solving the problme, some overhead is introduced because
multiple SMON peer packages may be copied, which wastes
network bandwidth and storage resources. Through the
evaluation, it is shown that the overhead is low on most
nodes. And the size of SMON installation package is small
(122KB), so the overhead can be considered as insignificant.

It is not easy to define when self-deployment is finished
globally. Ideally, it is finished when all the target
machines are reached and deployed with SMON peers.  However,
considering some nodes may be unreachable or shutdown
temporarily, the `finished-globally' situation is rare. Even
after a machine is deployed with SMON peer, it may be failed
and replaced with a new machine with same host name after
some time. The peers then have to continue the
detect-and-deploy process as long as they are live and leave
the `finish' definition to user. The user can query how many
or which peers are deployed, and determine whether it is
appropriate to consider the situation as a finish.

\subsection{Self-update}

As a distributed system, SMON can update itself to a new
version automatically. The update goes online while SMON is
running. Operator needs not to stop SMON before update.

Each SMON peer has a version number and it is stored
persistently in configuration file with the peer. When two
peers communicate using ping-pong messages, they also
piggyback their version numbers in the messages. If two
peers find a difference in their versions, the peer with
lower version will copy the installation package from the
peer with higher version and update itself automatically.

To update the whole SMON system to a new version, operator
only has to update one peer to the new version and all the
other SMON peers will emerge to the newest versions
eventually. The operator can update a SMON peer by copying
the SMON installation package of new version to local file
system of the peer's machine, and instructing the peer to
update itself by using a RPC call. The peer will run the
installation package, start the new peer and exit.

\note{need revise the method, consider the ping interval}
When a SMON peer knows there's a new version of peer, a list of
hosts on which peers with new version are running is gathered
through active detection among peers.  A self-update action is
scheduled to read the list and download the new version. At the
beginning of the self-update process, there would only be a
small number of peers running in new version, but they are known
by a lot of other peers.  If there's not any constraints, a lot
of peers may download from the same host simultaneously. To
prevent that, we use a conservative strategy to control the
self-update action.  The destination is, at average, only a
constant $k$ of peers can download simultaneously from a host.
The algorithm we use does not need any coordination among peers,
and works as follows.  If there are $n$ peers want to download
from the same host. Each of them choose to take the download
action  with probability $k/n$. If the action is not taken, it
will be delayed in the next time. So at average, there will be
only $k$ peers downloading from the same host simultaneously,
and it takes a peer to rerun the action $n/k$ times to select
the same peer.  As peers don't coordinate with each other, they
make a conservative estimation that set $n$ to $N$, the number
of all the peers. The strategy may cause a slow start at the
begin of self-update process because $n$ is far smaller that
$N$. When more peers are updated to new version, the update
process would be boosted. If a peer knows $p$ peers of new
version, the probability it choose to not download from any of
them is $q=(1-k/n)^p$ and it takes at average $\frac{1}{1-q}$
tries to choose a host. When $p$ increases, $q$ decreases
quickly and $\frac{1}{1-q}$ will be nearly 1 with large $p$.

\comment{
It is important to ensure that communication interfaces for
different versions of SMON peers are consistent. If the condition
is ensured, the whole SMON system can be updated gracefully while
most of peers are running and the system functions well.
Otherwise, the new SMON system can only be deployed after the old
one is stopped.
}

\subsection{Self-recovery}

Two kinds of failures may happen and must be handled by SMON,
namely, machine failure and network partitions.

Node failures are handled by self-deployment mechanism.
When a machine fails, the SMON peer running on it is stopped
also.  As soon as the machine recovers, it will be noticed
by some SMON peers quickly and the deployed SMON peer will
be restarted. Starting multiple instances of SMON peer on
the same machine is idempotent to start only one instance
and this is implemented using lock files. When Multiple SMON
peer instance is started, the first one will hold the lock
and the others will exit gracefully.  If the failed machine
is replaced with a new one, it will be installed by a new
copy of SMON peer.

Network partitions can also be handled by SMON easily. When
network partition occurs, SMON peers are splitted into
several sub-systems, each of which is connected internally.
Implied by epidemic algorithm, each partitioned part will
eventually emerges to the consistent state regarding to the
version of SMON peers. When two partitions rejoin, the peers
in different part will contact and communicate with each
other, leading to the emergence of the whole system.

\subsection{Security}
\label{subsec:security}

It is necessary for different SMON peers to authenticate
each others and keep the communication confidential among
them. In addition, when a SMON peer deploys another SMON
peer on a remote node, or a SMON peer copy installation
package from another peer for update, it need to
authenticate itself to the remote node to copy contents and
execute commands (start the installation package).  The
authentication credential is owned by user and should be
protected confidentially. From another point of view, the
SMON peers relies upon the authentication credentials
provided by user to deploy and manage each others
automatically in a distributed approach. 

SMON system is designed for distributed infrastructures that
are semi-open---like Planet-Lab or grid, or for proprietary
ones---such as large machine rooms or data centers. It
assumes that: 1) the network between nodes in a distributed
infrastructure is not trusted, 2) the users that can access
the infrastructures are not adversarial, 3) the applications
running in the infrastructures are not adversarial. Based on
assumptions above, SMON designed an security mechanism which
guarantees that 1) communications among SMON peers are
authenticated and confidential, 2) authentication
credentials needed to authenticate with remote nodes are not
leaked under any circumstances. The mechanism incorporated
in current design supports challenge based authentications
like public-key authentication scheme, which is extensively
used.

To achieve that goal, SMON includes an separate
authentication agent acting as the authentication bootstrap
for SMON.  Whenever a SMON peer needs to login into a node,
it will indirect the authentication challenges to the agent
and the agent will solve the challenges and send the result
back. In this way, the authentication credentials are kept
secret for all the times. A symmetric encryption key $K_E$
is shared among all the SMON peers and the authentication
agent. It is used to authenticate peers and agent within the
same SMON system, and establish confidential communication
channels among them. The shared key $K_E$ is included within
the installation package of the SMON peer. When a SMON peer
is deployed, $K_E$ is deployed also. Note that $K_E$ is
never leaked out of network during deployment. $K_E$ is safe
to be stored within local nodes based on assumption 2) and
3), and further local security mechanisms can be leveraged
to protect $K_E$, such as file system read/write
permissions.

It is not necessary to keep authentication agent running for
all the time.  When the self-deployment process is finished,
the agent can be closed, and re-open timely for maintennace
purpose.  It is also a security concern to offline
authentication agent to protect user authentication
credentials.

The authenticate agent can be also replicated if required.
When a SMON system is deployed at large enough scales, there
can be multiple instances of authenticate agent for
performance considerations. Name server based load balance
mechanism\cite{} can be used to direct SMON peers to their nearest
agents.

\subsection{Membership}

The membership maintenance of SMON system is different from
other distributed systems. While other distributed systems
maintains a list of peers that are currently running, SMON
maintains a list nodes on which there should be peers
running.  And an important requirement of SMON is, even when
there is only one SMON peer running, it should be able to
know all the target nodes that SMON need to be deployed. The
implication is every running SMON peer should be able to
know the full list of target nodes.

In current design, a simple scheme is used that each SMON
peer will store a replica of the full list of target
management nodes persistently in compressed format. The list
also has an associated version number and is maintained in
epidemic approach among peers. To further reduce the
communication overhead, only the differences between two
version are propagated during node list update process. This
scheme is enough for handling node list containing even tens
of thousands of nodes because only the node names or IPs are
stored in the list.  An estimation shows that, storing
15000+ nodes names in zip format takes about 100K bytes,
which is an acceptable result.

\subsection{Start and stop}

The SMON system has a global state indicating it is active
or not. When SMON is active, the peer-monitoring component
is enabled. Every SMON peers actively monitor each others
and keep them running as possible as they can. While SMON is
not active, the peer-monitoring function is disable and SMON
acts just like an usual management tools. The active state
is an essential part of SMON system design. Without it, SMON
is just another usual management tool for distributed
applications. If the state is designed as active all the
time, it is a hard problem to stop an active SMON system. If
we try to stop the SMON peers one-by-one, we will noticed
that the stopped peers will soon be restarted by other
peers. And the only way to stop an active SMON system is to
stop all the peers near simultaneously, which is quite hard.
The active state also has an associated version number and
each SMON peer replicates and stores the state with version
locally.  The epidemic algorithm is used to maintain the
state consistently and efficiently among all peers.

\subsection{Application management}

which bin to start
on failure: report, restart N times, ignore
watch pid file

NO app update, not necessary and too complex

One or more distributed applications can be managed by SMON. Each
SMON peer manages the application peers that run in the same
node. Each of the managed application peers has an associated
version number.  The version numbers of application peers are
maintained by managing SMON peers. The similar epidemic algorithm
is used for keeping application peers up-to-date. A trick used
in application management is, when an application peer has not
been deployed on a node yet, its version number is set to 0.0,
which is smaller than any real version numbers. In this way,
application deployment problem is turned to the application
update problem.

SMON peers knows the existence of a new application or a new
version of application through periodically communicating with each
other. A new application entry is created if it is known for the
first time.  The peers download the application with specified
version from other peers. There is a set of implemented RPC
calls for control and monitoring of managed applications.


\subsection{Client utility}

query SMON, live? version?

query app, live?

update version

install application

set/query livetag, 

maintain version number of SMON and live tag


The one-hop source routing\cite{Gummadi2004} algorithm is
implemented by SMON. When a peer failed to communicate directly
to another peer, it randomly selected $k$ ($k=4$ in current
design) other peers for sending the messages indirectly.  This
feature is useful for managing large number of nodes. We
implement this mechanism because we notice a lot of connection
problems caused by name resolution error. It happens between our
laptop and PlanetLab nodes, and also among some PlanetLab nodes.
In the experiment, we noticed a node cannot connect to other two
nodes because of name resolution error for more than one day.
But through one-hop source routing, they can communicate without
problem. This feature is useful for users to control or monitor
nodes at large scale. As the user control or monitor other nodes
through a peer as proxy, one-hop source routing can increase the
number of connected peers.

\subsection{Discussion}




% vim:foldmethod=marker:textwidth=60
