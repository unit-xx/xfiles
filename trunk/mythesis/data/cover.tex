% vim:textwidth=70:lines=42

%%% Local Variables:
%%% mode: latex
%%% TeX-master: t
%%% End:
\secretlevel{绝密} \secretyear{2100}

\ctitle{分布式系统运营准备中的若干关键问题研究}
% 根据自己的情况选,不用这样复杂
\makeatletter
\ifthu@bachelor\relax\else
  \ifthu@doctor
    \cdegree{工学博士}
  \else
    \ifthu@master
      \cdegree{工学硕士}
    \fi
  \fi
\fi
\makeatother


\cdepartment[计算机]{计算机科学与技术系}
\cmajor{计算机科学与技术}
\cauthor{高崇南} 
\csupervisor{郑纬民教授}
% 如果没有副指导老师或者联合指导老师,把下面两行相应的删除即可。
% 日期自动生成,如果你要自己写就改这个cdate
%\cdate{\CJKdigits{\the\year}年\CJKnumber{\the\month}月}
\cdate{\CJKdigits{2009}年\CJKnumber{11}月}

\etitle{Research on Key Problems of Distributed System Operational
Readiness}

% \edegree{Doctor of Science} 
\edegree{Doctor of Engineering} 
\emajor{Computer Science and Technology} 
\eauthor{Chongnan Gao} 
\esupervisor{Professor Weimin Zheng} 
% 这个日期也会自动生成,你要改么?
\edate{November, 2009}

% 定义中英文摘要和关键字
\begin{cabstract}

% 800-1000汉字

  分布式系统已成为支撑Internet服务的关键组成部分,随着系统规模越来越大
  、处理逻辑日趋复杂,在系统设计实现后,需要经过运营准备阶段对系统进行
  实际部署、测试、分析与调试。本文针对分布式系统运营准备中的若干关键问
  题进行了研究,取得了有价值的研究成果。主要贡献包括:

    1. 提出了使管理系统具备自管理机制的方法,设计实现了自管理的
    分布式管理系统SMON。SMON的自管理方法基于epidemic算法和认证代理技术,
    保证了可扩展性和安全性。理论分析表明,SMON的自管理操作时间
    随系统规模呈$O(\log N)$增长,自我部署的额外负载和节点收到维护消
    息频率是常数($O(1)$)。在Planet-Lab平台上的实际评测证实了理论分
    析的正确性。

    2. 研究了分布式分发优化的策略选择问题。通过在真实Internet测量数据
    上的广泛细致评测,研究了邻近邻居选择与动态上传带宽分配两种优化方法。
    得到了如下重要结论:临近邻居优化效果随邻近邻居比率增加而单调增加,
    且依赖于LRF块选择算法;使用贪心算法确定的上传节点集合使优化效果达
    到局部最优,且不依赖LRF块选择算法;合并两种优化方法不具有叠加优化
    效果。

    3. 提出了自动推断系统层次结构任务模型的方法。解决了已有工作需
    要需要手动标注任务边界与依赖关系的问题。本方法使用同步点自
    动推断任务边界,使用happened-before关系自动推断任务依
    赖关系,使用图上的聚类方法自动推断任务层次结构。在实际系统上
    的应用表明,推断出的任务模型能帮助理解与验证系统设计,调试系统性
    能问题。帮助调试了PacificA分布式存储系统(类似BigTable)
    的性能问题,该问题使压力测试中网络带宽利用率远低于100\%。

    4. 提出了从系统日志中推断任务层次结构的方法。系统日志代表了系统运
    行时的活动,本推断方法能够自动从无结构的系统日志文本中提取任务信息,
    并在一组任务上推断任务之间的层次结构关系。使用推断的任务层次关系,
    可以帮助理解与验证系统设计,解决系统性能问题。帮助解决了ChunkFS(
    类似GFS)的性能问题,该问题使压力测试中网络带宽利用率远低于100\%。


\end{cabstract}

\ckeywords{分布式系统; 自管理; 数据分发; 层次结构任务模型; 日志分析}

\begin{eabstract}

  Distributed systems are the key components in today's Internet
  services. While the system scale goes larger and the design logic
  becomes more complex, it is a challenge to managing and debugging
  large scale distributed systems at operational readiness stage.
  While current research work makes great progresses, some key
  problems are still not fully addressed yet. In this thesis, we
  conducted research on self-management of distributed systems and
  automatica inference of hierarchical task models of distributed
  systems.

  The contributions of this thesis are:

  \begin{enumerate}

    \item A mechanism to support self-management of distributed
    application management system is proposed and a self-managed
    distributed application management system (SMON) is designed. SMON
    design is based on epidemic algorithm and authentication agent.
    SMON can automatically and securely deploy itself to a set of
    machines, recover itself from failures and upgrade itself to newer
    versions online. Theoretical analysis shows that the
    self-management time grows $O(\log N)$ with system scale. The
    overhead number in self-deployment and the frequency of received
    maintenance messages are both constant ($O(1)$).  SMON supports
    management semantic for long running internet servies and can be
    extended easily.

    \item Emperical study on performance optimization of distributed
    data distribution is presented. Based on real Internet
    measurements, the effect of two optimization methods---biased
    neighbour selection and dynamic upload bandwidth allocation---is
    studies extensively. Several important conclusions are made: the
    optimization effect of biased neighbour selection grows
    monotonously with ratio of biased neighbours, and relies on LRF data
    block selection strategy; the upload nodes set determined by
    greedy strategy in dynamic upload bandwidth allocation achieves
    local optimum, and the optimization doesn't rely on LRF data
    block selection strategy; combining the two optimization
    strategies achieve marginal effect.

    \item An automatic inference methodology of hierarchical task
    models for distributed systems is developed. The inference method
    can automatically infer task boundaries using synchronization
    point as heuristic, correctly associate task dependencies using
    happened-before relation as heuristic and infer task hierarchies
    using graph mining algorithm.  The hierarchical task models help
    on understanding of system design and implementation, and help on
    finding root causes of system correctness and performance bugs.
    A performance bug in Pacifica distributed storage system (similar
    with BitTable) is resolved using the inferred task model. The
    bug causes that the bandwidth cannot be saturated in stress test.

    \item Methodology of inference of hierarchical task models based
    on system log is proposed. The log presents system runtime
    activities and behaviours. The inference methodology can extract
    task information from unstructured system logs and infer
    hierarchical relations among the tasks automatically. The
    hierarchical task models help on understanding of system design
    and implementation, and help on finding root cause of system
    correctness and performance bugs. A performance bug in ChunkFS
    distributed storage system (similar with GFS) is resolved using
    the inferred task model. The bug causes that the bandwidth cannot
    be saturated in stress test.

  \end{enumerate}

\end{eabstract}

\ekeywords{Distribute System; Self-Management; Data Distribution;
Hierarchical Task Model; Log Analysis}
