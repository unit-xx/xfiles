% vim:textwidth=70:lines=42

%%% Local Variables:
%%% mode: latex
%%% TeX-master: t
%%% End:
\secretlevel{绝密} \secretyear{2100}

\ctitle{分布式系统管理与调试若干关键问题研究}
% 根据自己的情况选,不用这样复杂
\makeatletter
\ifthu@bachelor\relax\else
  \ifthu@doctor
    \cdegree{工学博士}
  \else
    \ifthu@master
      \cdegree{工学硕士}
    \fi
  \fi
\fi
\makeatother


\cdepartment[计算机]{计算机科学与技术系}
\cmajor{计算机科学与技术}
\cauthor{高崇南} 
\csupervisor{郑纬民教授}
% 如果没有副指导老师或者联合指导老师,把下面两行相应的删除即可。
% 日期自动生成,如果你要自己写就改这个cdate
%\cdate{\CJKdigits{\the\year}年\CJKnumber{\the\month}月}
\cdate{\CJKdigits{2009}年\CJKnumber{11}月}

\etitle{Research on Key Problems of Distributed System Management and Debugging}

% \edegree{Doctor of Science} 
\edegree{Doctor of Engineering} 
\emajor{Computer Science and Technology} 
\eauthor{Chongnan Gao} 
\esupervisor{Professor Weimin Zheng} 
% 这个日期也会自动生成,你要改么?
\edate{November, 2009}

% 定义中英文摘要和关键字
\begin{cabstract}

% 800-1000汉字

  分布式系统成为了支撑Internet服务的关键组成部分,随着系统规模越来越大,
  系统处理逻辑日趋复杂,有效的管理与调试分布式系统成为了挑战。目前相关
  研究取得了一些进展,但很多关键问题尚待解决。本文针对分布式系统自管理
  机制、分布式数据分发系统优化、自动推断分布式系统层次结构任务模型的方
  法等方面进行了研究,取得了有价值的研究成果。

  本文主要贡献包括:

  \begin{enumerate}

    \item 提出了使管理系统具备自管理机制的方法,此方法基于epidemic算
    法和认证代理技术。设计并实现了一个自管理的分布式应用管理系统SMON。
    SMON能够自动将自己安全部署到一组指定的机器上去,将自己从运行错误中
    恢复,并在线升级自己至新版本。SMON支持长期运行的Internet服务管理语
    意,用户可以通过在SMON上部署新的管理系统扩展管理语意。

    \item 研究了分布式分发系统的优化问题。通过在真实Internet测量数据上
    的广泛细致评测,研究了邻近邻居优化和动态上传带宽分配两种优化方法的
    特性、优化方法受数据块选择算法的影响和合并两种优化方法的效果。对实
    际系统中应用优化方法提出了建议。

    \item 提出了一自动推断系统层次结构任务模型的方法。推断方法在系统运
    行时trace上,自动推断任务边界、任务的因果依赖关系和任务的层次结构。
    使用得到的层次结构任务模型,可以对系统设计有深入的了解,并帮助分析
    导致系统缺陷的原因。

    \item 提出了从系统日志中推断任务层次结构的方法。推断方法能够自动从
    无结构的系统日志文本中提取任务信息,并推断任务之间的层次结构关系。
    使用得到的任务层次关系,可以帮助理解系统设计,解决系统已有的性能问
    题。

  \end{enumerate}

\end{cabstract}

\ckeywords{分布式系统; 自管理; 数据分发; 层次结构任务模型; 日志分析}

\begin{eabstract}
  Distributed systems are the key components in today's Internet
  services. While the system scale goes larger and the design
  logic becomes more complex, it is a challenge to managing and
  debugging large scale distributed systems. While current research
  work makes great progresses, some key problems are still not fully
  addressed yet. In this thesis, we conducted research on
  self-management of distributed systems and automatica inference
  of hierarchical task models of distributed systems.

  The contributions of this thesis are:

  \begin{enumerate}

    \item A mechanism to support self-management of distributed
    application management system is proposed and a self-managed
    distributed application management system (SMON) is designed. SMON
    can automatically and securely deploy itself to a set of machines,
    recover itself from failures and upgrade itself to newer versions
    online. SMON supports management semantic for long running
    internet servies and can be extended easily.

    \item Emperical study on performance optimization of distributed
    data distribution is presented. Based on real Internet
    measurements, the effect of two optimization method---biased
    neighbour selection and dynamic upload bandwidth allocation---is
    studies extensively. The interaction between the optimization
    methods and the piece selection method, and the effect of
    combining two optimization methods is studied.

    \item An automatic inference methodology of hierarchical task
    models for distributed systems is developed. The inference method
    can automatically infer task boundaries, correctly associate task
    dependencies and infer task hierarchies.  The hierarchical task
    models help on understanding of system design and implementation,
    and help on finding root causes of system correctness and
    performance bugs.

    \item Methodology of inference of hierarchical task models based
    on system log is proposed. The inference methodology can extract
    task information from unstructured system logs automatically and
    infer hierarchical relations among the tasks. The hierarchical
    task models help on understanding of system design and
    implementation, and help on finding root cause of system
    correctness and performance bugs.

  \end{enumerate}

\end{eabstract}

\ekeywords{Distribute System; Self-Management; Data Distribution;
Hierarchical Task Model; Log Analysis}
