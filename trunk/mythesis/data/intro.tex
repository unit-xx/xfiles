% vim:textwidth=70
\chapter{引言}
\label{chap:intro}

\section{选题背景}
% 管理和任务模型怎么捏到一起

% 分布式系统需要管理->管理=部署+维护(=监测+恢复)

% SMON维护应用的方法(online、offline、ignore)只是一种维护机制,可以
% 用于跑实验或长期服务的维护。维护机制可以有其他选择,例如可以没有(就
% 是ignore),可以用scalpel动态抓任务模型,看有没有正确性或者性能问题。

% 维护=监测+动作
% 监测:进程死活,应用内部状态等待
% 动作:重启?获取任务模型等等。

Internet服务正在改变着人们的生活方式。搜索引擎改变了已有知识的组织与访
问方式,它使得全世界的人们,可以通过互联网方便地获取感兴趣的内容,这包
括在线文档、图片、视频等内容,也包括更专业的文献、新闻、金融信息、博客
等搜索服务。除了搜索引擎,在线邮件服务让人们通过PC或者手机就可以随时随
地处理自己的邮件。在线文档服务使得人们不必安装专业软件,就可以创建、编
辑与共享信息。同时,多个个体可以协同编辑同一文档,而不必考虑底层的存储
维护、一致性保证等工作。在线交友服务让人们有了新的互相认识与交流的方式,
它扩大了人们相互认识的途径,也改变了老朋友们维持友情的方式。

%其它internet服务还有在线视频、照片。在线视频,交友,照片,博客。影响了
%其它产业,例如手机

支撑这些internet服务的是一些大规模的分布式系统。分布式存储系统可靠的保
存着回答搜索请求所需要的巨大容量的信息,例如网页、图片、视频、地理信息
、金融数据等。从而保证了在任何时间、任何地点的人们,都可以有效地查询信
息。内容分发网络(CDN,Content Distribution Network)将网络服务“推送”
到离用户最近的网络边缘,这样,用户就能够就近快速访问服务,同时减轻了服
务提供者的负载。基于P2P覆盖网络的系统被广泛的应用在文件分发、共享,视
频组播等服务。P2P覆盖网络让用户能够相互共享与分发数据,从而有效提高了
网络资源的利用率,同时减轻了服务器的压力。在学术界与工业界,人们也在尝
试使用P2P技术,搭建具有高度可扩展性和稳定性的存储系统,提供“云存储”
(Cloud Storage)服务。

早期的internet服务基于单机系统。世界上最早的搜索引擎Archie起初只使用了
一台机器,它定期从一组FTP服务器获取文件列表,为用户提供文件查询服务。
随着用户越来越多,Archie逐渐从单机系统,演变成由前端和后端组成,在
internet上有多处服务器的分布式系统。

随着internet服务的普及与流行,支撑它们的分布式系统规模变的越来越大。以
Google为例,2006年的数据\footnote{Google有意保密其系统规模等参数,因此
难以获得最新数据}表明,支撑Google服务的集群规模已经达到了450000台机器,
其耗电量达到了20兆瓦特,平均每月电费在200万美元的规模。存储在这些集群
上的数据容量也很大,在2008年召开的Google I/O会议上,Google Fellow
Jeffrey Dean披露说,最大的BigTable运行实例包含了超过6 petabytes的数据。

% 现在,cluster规模: 450,000, 2006,how google works, wiki:google
% platform

有若干因素使得internet服务必然选择分布式系统作为关键组成部分。首先,
internet服务的成功,吸引了众多使用者。服务不再可能使用单机或者小规模集
群就能够胜任。随着普通计算机元件造价越来越低,人们可以很容易的搭建出规
模上千的分布式系统。用户请求被分散到不同的机器并行处理,提高了服务的吞
吐率与处理延迟。服务还会被分布到不同地区,使用户能够就近访问,减小了服
务的处理延迟。

其次,是为了保证服务的可靠性与可用性。用户希望服务总是可用的,也就是服
务总是可以访问的。同时,服务也应该是可靠的,用户的数据不应该丢失。因此,
系统使用冗余机制,将数据与服务复制多份。即便遇到故障,仍然能够处理用户
请求。

最后,有些internet应用,其本身就必须是分布式系统。例如,基于P2P的文件
共享,视频组播等。只有分布式系统,能够将世界各地的用户连接到一起。

\note{上面的内容还可以再扯一扯}

随着分布式系统被广泛应用,它也迅速成为了工业界与学术界的关注热点。各种
各样的分布式系统被提出并研究。我们注意到,分布式系统不仅难于设计,同时,
如何有效地管理已经设计与实现的分布式系统,如何调试分布式系统的正确性与
性能缺陷,也是非常重要的问题。

与传统单机软件不同,分布式系统需要被有效地管理。分布式系统由运行在不同
机器上的应用节点组成,应用节点通过网络相互连接,共同完成设计的功能。因
此,需要首先将系统部署到一组机器上才能运行。这其中的核心问题是如何将应
用分发到规模成百上千台机器上去,考虑到系统规模巨大,同时各种故障随着规
模增大而频繁发生(例如机器故障、网络连接断开),因此部署一个分布式应用
并不容易。在应用部署并启动之后,它的运行状态需要被紧密监测,以保证系统
运行正确、稳定,性能达标。

分布式系统的调试也是个挑战。规模大,并行处理,请求经过不同节点,异步处
理,更复杂的逻辑,错误处理。(或者应该说调试研究什么,而不是研究的难
点。)

本文分别研究了分布式系统管理与调试中的关键问题。具体的,针对xxx我们研
究了xxx。

% 从历史看分布式系统发展的脉络,从应用看分布式系统使用的必然性

% 随着系统规模越来越大,设计一个分布式系统遇到了新的问题与挑战。


% 与单机运行软件不同,一个分布式系统在设计与实现之后,如何管理与调试也面
% 临着诸多挑战。

% 1. motivate 管理和调试
% 2. 管理和调试捏到一起
% 3. current limitation

\subsection{Internet服务与分布式系统}

这章或许可以不要,等写完了下面章节的内容再回来看看。

这章其实就是扯淡,主要述说了internet服务中的分布式系统都有什么,各有什
么特点。

\subsection{分布式系统设计}

设计一个分布式系统需要解决许多困难问题,通常下面这问题是必须考虑的:

可扩展性;可靠性;容错;一致性;可扩展性

\section{本文研究内容}

\subsection{分布式系统管理}

分布式系统或分布式应用\footnote{在后面的叙述中,我们交叉使用“分布式系
统”或“分布式应用”来表达相同的意思。}管理包括若干问题。分布式系统需
要运行在由网络连接的一组机器上,不同的系统对机器、网络资源配置的要求也
不同。我们需要在分布式计算平台提供的资源基础上,选择出一个符合要求的子
集,这就是\emph{资源发现}。其次,需要将分布式系统\emph{部署}到这组机器
上,包括分发系统的安装程序,在每个机器上配置运行参数,并在各个机器启动
系统运行。在系统运行期间,需要\emph{监测}系统运行状态。这包括对系统所
在计算平台的监测,也包括对系统本身状态的监测。最后,管理者希望能够动态
的\emph{控制}系统的运行,包括恢复失败的系统进程等。

\subsubsection*{资源发现}

支撑分布式系统运行的分布式计算平台规模可能非常大,Planet-Lab包括了超过
900个节点,而Google使用的集群规模达到了数万台机器。通常一个分布式系统
并不需要使用计算平台上的所有机器。开发者或者管理者需要找到满足系统运行
的一组资源,包括一组机器以及连接这些机器的网络。

不同的分布式系统对资源的要求不同。科学计算程序希望能够找到空闲CPU、内
存都很大的机器,如果计算过程中有频繁的数据交换,那么这些机器之间的网络
连接状况也应该很好。一些科学计算程序运行时间较短,从几小时到几天。因此
它们更关心资源的当前使用情况。还有一些分布式系统提供长期的服务,例如资
源发现服务本身。它们定期收集计算平台的资源使用情况,并执行来自用户的查
询。这些服务并不需要使用很多CPU与内存资源,但是它们希望运行在稳定性高
的一组机器上,频繁的机器或网络故障会影响服务的质量。

资源发现服务接受用户对所期望资源的描述,并返回满足要求的一组资源,如果
没有资源能够满足用户的期望,则返回空。抽象来说,这是一个图的匹配问题。
资源发现服务定期的收集整个分布式计算平台的资源使用数据,包括每个机器的
计算资源和存储资源,以及机器之间网络连接情况,将其以适合的方式保存。用
户将对期望资源的描述用图的形式表达。图的节点描述了机器应当满足的条件,
图的边描述了机器之间的网络连接应当满足的条件。资源发现服务要在整个计算
平台这个大图上,寻找满足用户条件子图,这等价于k-clique问题,是一个
NP-hard问题。

\subsubsection*{部署}

部署是将分布式系统分发、安装到选定的一组机器上,并配置、启动系统运行的
过程。其核心是如何将系统分发复制到一组机器上。

可以使用集中式的方法部署分布式应用。管理者从一个集中控制机器,将系统的
安装程序远程复制到各个机器上,并逐个配置并启动应用的进程。这种方法最大
的优点是设计实现简单,许多情况下,只需要编写几个脚本就可以完成绝大多数
工作。

集中式的方法对小规模应用很有效,但是却不能很好的胜任大规模时的情况。首
先它的可扩展性很差,在部署多台机器时,所有的数据都通过本地网络分发,效
率低下。其次,在大规模情况下,网络与机器故障成为频繁发生的事情。集中式
方法使用的静态星型拓扑结构不能很好的处理这些故障。

因此,各种基于P2P算法的部署工具被研究并开发。这些工具本身也是一个分布
式系统。工具的节点分布式在各个机器,节点通过相互合作,从对方获取自己没
有的数据块,从而有效地节约了数据源所在节点的网络资源。同时,这些工具能
够动态的适应网络环境的变化,选择最佳的数据传输路径,从而提高了部署的效
率,也能有效地应对一些网络故障。依据节点间形成的网络拓扑结构,可以将工
具分为基于随机结构网络拓扑的,或者基于结构化的网络拓扑的。

一些复杂的分布式应用通常由多种类型的节点构成。例如一个典型的三层结构的
Web应用,由前端的Web服务器,中间的应用处理和后端的数据存储构成。每一层
都可能是一个分布式系统。为了有效地支持这种复杂的部署需求,一些功能上更
先进的部署工具被研究并开发出来。这些工具提供了与自身绑定的描述语言,使
用者使用这些语言描述部署说明,使得工具能够自动完成指定的部署任务
(smartfrog, plush)。

\subsubsection*{监测}

分布式系统的运行状态散布在各个机器,人们需要监测系统的运行状态。这包括
底层计算平台的状态,与应用本身的状态。所监测的状态指标有表示系统是否正
确运行的,也有表示系统运行性能的。

在许多情况下,监测的目标并不是针对每个机器或者每个应用进程,而是针对所
有状态的一个聚集运算。例如,返回所有CPU使用率超过95\%的前十个机器列表。
最直接的办法是将所有状态集中收集到一起,然后进行计算。然而这样的方式扩
展性太差,在实际中没有可操作性。一些算法使用树的结构,将需要聚集的属性
从树叶向上传递,树中的节点将子节点的结果聚集后再向上传递,因此减少了数
据的传输。还有一些系统使用基于epidemic和gossip的算法,通过节点之间随机
的交换数据,达到统计意义上保证一定准确率的聚集结果。

使用聚集的方式可以持续监测某个全局状态的值,但是在一些情况下,人们只想
在知道全局状态是否满足一些约束谓词。例如,从每个机器访问任意站点的流量
总和不能超过100M/s,这是监测DDoS攻击的一个全局约束。在这种情况下,我们
不需要持续的聚集每台机器的流量状况。针对这种需求,可以使用分布式触发器
(distributed trigger)技术。这种技术在约束没有被违反的情况下并不传输
任何数据,因此进一步降低了检测所带来的流量负载。

\subsection{分布式系统调试}

分布式系统设计复杂,是为了满足可扩展性等要求。对调试有了新的困难。

基于对分布式系统运行状态检测的分析。


\section{本文研究内容与贡献}

\subsection{研究内容}

\subsubsection*{可自管理的分布式应用管理系统}

为了有效地管理分布式应用,其管理系统也变得越来越复杂。管理系统本身也成
为了一个复杂的分布式应用,由散布在许多机器上的管理节点构成,共同完成应
用的部署、监测与控制等任务。

这就带来一个问题,也就是,由于管理系统变得越来越复杂,其本身也需要一定
的管理工作。需要部署管理系统的节点,并监测其是否正常运行。使用现有的技
术,有两种解决这个问题的方法。其一、使用集中式方法管理需要使用的管理系
统。集中式方法易于实现,其本身也易于被管理。但是集中式算法扩展性差,对
网络资源的利用效率低,同时不能够很好的应对分布式环境下经常发生的网络异
常。其二、使用一个采用P2P算法的管理系统去管理我们需要使用的管理系统。
这个方法虽然表面看上去很有效,但是新采用的管理系统又递归的产生了需要被
管理的问题,所以这个方案并没有解决问题。

在本文中,我们通过给管理系统增加内建的自管理能力,从根本上解决了管理系
统本身也需要被管理的问题。我们设计了一个具备自管理能力的分布管理系统
SMON(Self-Managed Overlay Network)。构成这个系统的节点相互监测运行状
态,并相互管理维护。SMON能自动将自己部署到一组指定的节点上去,也能够自
动恢复运行失败的节点,同时能够在线将整个系统更新至新的版本。

设计SMON有如下的几个难题。首先是保证其自管理具有良好的可扩展性,其次,
为了支持自动部署,需要一定的安全机制保证系统节点能够自动与计算平台的机
器认证并登陆,完成远程部署的任务。最后,在管理功能上,SMON应该具有良好
的可扩展性。

我们分别解决了这几个难题。首先,我们使用epidemic算法实现自管理。SMON节
点相互随机探测,并依据探测的结果,自动在远程机器上部署新的SMON节点,或
者恢复失败的SMON节点,或者更新自身至新的版本。其次,我们使用了认证代理
使SMON节点能够自动与计算平台的机器认证并登陆。SMON节点将登陆认证过程中
的挑战密文转发给认证代理,并返回认证代理求解的应答密文给远程机器,从而
完成自动认证的过程。同时,这一机制保证了用户提供的认证信息不会泄露到认
证代理外部。最后,由于SMON具有自管理功能,因此任何被SMON管理的分布式应
用也一并具有了自管理的能力。进而,可以在SMON上部署其它的分布式应用管理
系统,从而扩充了整个系统的管理能力。

我们在Planet-Lab平台上对SMON进行了评测,结果显示,SMON具有很好的扩展性
和性能。

\subsubsection*{自动推断系统层次结构任务模型的方法}



\subsubsection*{基于日志的系统任务模型推断方法}

\subsection{目标使用者}

本文针对分布式系统管理与调试的一些关键问题进行了一些研究,其研究结果可
以被一下三类使用者使用:

\begin{description}

\item[系统开发者] Primary programmers: the original authors of a system, who
wish to debug or optimize their own code. They have access to
application source code and are familiar enough with it to know what
constitutes expected behavior.

\item[系统代码维护者] Secondary programmers: other maintainers or
contributors to an application, who have access to the source code but
might be unfamiliar with it. For example, secondary programmers might
be those who have inherited or joined an existing project. Their first
goal is to learn about an application, and they might not know what
system behavior is expected.

\item[系统管理员] Operators: programmers or system administrators who
must keep a production system correct through upgrades and other
configuration changes. They often do not have access to source code,
but can still use black-box techniques or existing source code
annotations to monitor an application's performance and correctness.

\end{description}

\subsection{研究贡献}

\subsection{内容组织}

