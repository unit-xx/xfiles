% vim:textwidth=70
\chapter{自动推断系统层次结构任务模型的方法}
\label{chap:scalpel}

\section{本章引言}

随着Internet服务逐渐渗入人们日常的生产生活,它们的可靠性也变的越来
越重要。系统设计实现上的缺陷,一直伴随着这些服务而生,导致服务性能下降,
甚至彻底中断运行。在诸多的软件缺陷(bug)中,那些最难找到和解决的是让
系统仍然运行,但是却偏离了系统期望行为的缺陷。这些缺陷的根本原因,隐藏
在复杂甚至是混乱的应用逻辑中,因此寻找并分析这些缺陷变的异常困难。

支撑Internet服务的系统本质上就非常复杂,这更增加了分析理解系统异常行为
的难度。这些系统通常使用了分层的体系结构,将其功能抽象表达为不同的层次
结构。\note{high degrees of concurrency}。系统在运行时,处理着许多用户
层次的\emph{任务},例如用户请求,任务被分成许多阶段执行,不同的阶段
被分布在多个机器、进程和线程上执行,使用事件或者异步消息作为通知机制。
验证单独每个任务的行为,是一件具有挑战性的问题,因为开发人员需要重构出
任务的执行流,将任务执行过程中的各个阶段重新连接起来。

从概念上说,开发人员可以把任务执行表示为\emph{层次结构的任务模型},
这与系统的分层体系结构设计一致。不同层次的任务表示在不同阶段不同模块执
行的一段执行过程。高层任务的执行被分为若干低层子任务。\note{pacifica
example}。基于任务模型,开发人员可以更好的理解系统模块间的结构,以及不
同模块间的依赖关系,并验证出于不同层次任务的行为。

然而,目前的\note{工具}需要开发人员手动标注任务模型。例如,
Pip~\cite{pip}要求开发人员将系统预期行为,用“期望(expectations)”的
形式表达,“期望”表达了系统正确执行时的任务模型,包括对执行时资源使用
的约束。通过对比期望与实际执行的区别,可以验证系统运行时行为。写出一个
全面表达系统高层设计与底层实现的期望是非常困难的并容易出错的事情,特别
对那些正在快速发展中的系统。基于执行路径的工具,例如
Magpie~\cite{magpie},可以从单独的运行事件trace推断每个请求的执行路径,
但是它仅可以处理的一组事前确定的事件,并且需要开发人员指定任务的边界与
关联条件。

本文的目的是探索不需要人工帮助,自动推断层次结构任务模型的方法。开发人
员不需要手动标注源代码来指定任务边界,并却任务的层次结构也应该能够自动
推断得到。这样,开发人员和系统管理员就可以利用得到的任务模型,以可视化
的形式表现系统设计和实现,也可以将任务模型作为输入,使用其它工具调试或
验证系统设计。

设计一个自动任务模型的推断工具面临如下几个挑战性问题。首先,应该能够确
认合适的任务边界,这一过程应该只基于对系统执行过程的监视,而不需要开发
人员显示的标注。其次,必须能够正确的关联任务之间依赖关系。特别是,必须
能够辨别任务之间因为共享资源(例如,共享队列、锁等)而产生的依赖关系。
最后,应该能够自动恢复任务的层次结构。任务由许多或者顺序或者并行的子任
务构成。考虑到复杂系统中任务执行的非确定性,确定它们的依赖关系与层次结
构并不容易。

在本文中,我们描述了如何自动推断复杂系统层次结构任务模型的方法,并实现
了一个原型系统Scalpel。我们通过使用插装(instrument)技术来透明的观测
系统运行,获取系统运行过程的trace,包括用户和系统函数调用的过程。自动
推断方法使用trace作为输入,自下而上(bottom-up)的推断系统层次结构任务
模型,推断的过程分为三步,分别对应上面提到的三个挑战性问题。

首先,我们将运行过程分为一个个叶子任务,它们是层次结构任务模型中最基本
的任务。叶子任务的边界对应执行的同步点(synchronization point)。在同
步点,线程或进程相互同步从而具有依赖关系,因此同步点是推断任务边界的一
个合理启发点(heuristic)。

其次,我们按照运行时的因果依赖关系,将叶子任务用有向边连接,形成一个因
果关系图。叶子任务的因果依赖关系由任务运行时的执行顺序推断得到。

最后,我们在任务关系图上,推断出层次结构。每一个层次,大致对应系统设计
与实现上的一个层次。我们使用聚类算法寻找任务关系图上重复出现的模式(就
是频繁子图),以此为依据确认高层任务与构成它的子任务。通过递归的使用聚
类算法,我们能够进一步寻找更高层次的任务。

\note{prelimianry experience}

\note{文章安排}


\section{设计}

本章叙述推断方法设计,以及它的原型实现\pozhehao{}Scalpel。

\subsection{收集系统运行trace}
收集什么

全序关系?



\subsection{确定叶子任务}

在系统执行trace的基础上,我们首先需要确定叶子任务的边界。Scalpel使用
\emph{同步点}作为启发点定义任务边界。同步点是两个线程同步相互执行,从
而具有执行顺序关系的地方。在同步点,线程可能因为互斥或者相互协同而具有
顺序关系,前者的例子是线程使用锁同步对共享资源的访问,后者的例子是线程
等待另外一个线程的信号消息。

我们定义两个相继的同步点之间的执行为一个叶子任务。采用这个定义的原因是,
首先两个同步点之间的一段执行是相对独立,并且自包含的。因此,它们合理的
成为任务模型中最小粒度的一段执行,也是叶子任务的一个自然定义。另外,因
为采用了同步点定义叶子任务的边界,叶子任务之间的依赖关系也只发生在边界
之间。

Scalpel通过插装系统函数库中的同步原语(锁、信号、事件等)与socket操作
\footnote{我们认为通信也是一种同步操作。}获得相关的trace。在实际系统中
的应用经验表明,插装这些系统操作是足够的。如果应用系统使用spin-lock或
者lock-free的数据结构,则仅插装系统调用是不够的。这时Scalpel会丢失一些
同步点从而使任务模型粒度更粗。手工标注可以解决这个问题,但是整个推断方
法不再是全自动的。鉴于多数实际系统还是采用系统调用进行同步,我们将这个
问题留作将来的工作。

\subsection{任务因果关系图}

为了推断任务的层次结构,我们首先要连接叶子任务直接的依赖关系。我们使用
因果关系图。这个图中,节点表示叶子任务,有向边表示叶子任务直接的依赖关
系。\note{For example, in PacificA...}。

Scalpel使用执行顺序关系推断任务的因果依赖关系,这包括同一线程内顺序执
行的两个叶子任务,以及因为同步而依次执行的叶子任务。我们需要将“假”的
因果依赖关系区别并去除。一个“真”的依赖关系的例子是,一个从队列里取出
并处理事件的任务,在因果关系上依赖于产生那个事件并将其加入队列的任务。
另一方面,如果两个线程使用互斥锁同步对共享资源(例如I/O)的访问,虽然
它们的行为在表面上也构成因果依赖关系,但实际上,任务直接并不相互依赖,
其执行的顺序有调度器随机决定。因此,我们将互斥排除在因果关系之外。

Scalpel使用若干启发点分辨真的因果依赖关系。如果系统使用操作系统提供的
队列(例如I/O completion ports),或者通知机制(例如event),则可以使
用同步操作使用的句柄(保存在插装得到的trace中,是同步操作的参数),将
生产线程与消费线程联系起来。对于操作系统的互斥锁(mutex)与信号量
(semaphore)对象,它们通常被用来同步线程对共享资源的访问,因此不认为
它们构成因果依赖关系。

\note{match socket}


\subsection{推断任务层次结构}

递归寻找frequent subgraph

具体做法

子图相似度定义

具体算法(伪代码)

\subsection{实现}

a

\section{讨论}

a

\section{验证推断方法}

iocp, event, mutex

\section{应用实例}

apache, sqlite

pacifica

// statistics

\section{性能评测}

overhead, script run time, 


\section{相关工作}

a

\section{本章小结}

a

