\section{Implementation}
\label{sec:impl}

We implement the prototype of SMON system in Python, and the
final installation package size of the SMON peer is about
122KB, with 1000+ lines of code.

%\begin{table*}
%\small
%\centering
%\begin{tabular}{|l|l|}
%
%\hline
%\textbf{RPC} & \textbf{Description} \\
%
%\hline
%\texttt{ping(ver)} & Call with local SMON peer's version and
%return remote peer's version.\\
%
%\hline
%\texttt{retrieve\_peer(ver)} & Retrieve installation package
%of SMON peer with specified version.\\
%
%\hline
%\texttt{exchange\_livetag(tag, ver)} & Call with local
%$<$livetag, version$>$, and return remote peer's $<$livetag,
%version$>$.\\
%
%\hline
%\texttt{exchange\_member(ver)} & Call with local membership
%list version and return remote peer's membership list.\\
%
%\hline
%\texttt{retreive\_member(ver)} & Retrive membership list of
%specified version.\\
%
%\hline
%\texttt{exchange\_app(app\_name)} & Call with an application
%name, return true if remote peer has installed the
%application.\\
%
%\hline
%\texttt{retrieve\_app(app\_name)} & Retrieve installation package
%of an application with specified name.\\
%
%\hline
%\texttt{resolve\_challenge(challenge)} & Return the response
%to an authentication challenge.\\
%
%\hline
%\texttt{set\_app\_status(app\_name, status)} & Set application status (online,
%offline, ignore).\\
%
%\hline
%\texttt{get\_app\_status(app\_name)} & Get application status. \\
%
%\hline
%
%\end{tabular}
%\caption{RPC interfaces implemented by SMON peer and authentication
%agent}
%\label{fig:rpc}
%\end{table*}

The maintenance tasks of a SMON peer is implemented in
several threads, including one for monitoring and
maintaining other peers, one for updating membership list,
one for update \texttt{livetag}. For each application, a
separate thread will be started for maintaining the
application.

The communication interface of SMON peers and agents are
implemented as XML-RPCs, which are not listed for space
limitation.

SMON peer uses paramiko\footnote{a ssh library implemented
in Python} to copy files or send commands to remote machines
using ssh protocol. It will redirect the authentication
challenge to the authentication agent using
\texttt{resolve\_challenge} XML-RPC interface.

The configuration parameters used at SMON runtime is stored
in a configuration file. It stores the time intervals among
consecutive maintenance tasks, the version numbers for SMON
peer and membership list, the $<$livetag, version$>$ tuple,
the address of authentication agent. The configuration file
is implemented as a SQLite database file to avoid data loss
at machine crash. The membership list is stored separately
as a compressed file.

%SMON makes a predefined and consistent directory layout of
%peer installation and it is critical for peers to perform
%their functions correctly. A SMON peer is installed to
%\texttt{SMON} directory of user's home directory, and this
%setting can be changed. Under \texttt{bin} directory, a
%\texttt{run.py} and \texttt{smon.cfg} is placed. To start
%SMON peer, \texttt{run.py} is executed and it read
%configuration in \texttt{smon.cfg} and start the version
%specified in \texttt{smon.cfg}. A new version SMON peer
%should change \texttt{smon.cfg} appropriately. Different
%version of SMON peer is placed according to version number
%under different directories. \texttt{app} directory contains
%different application entries, as different directories. A
%configuration file for each application is placed under it
%directory. Installation package of SMON peer and
%applications are stored under \texttt{packages} directory,
%placed according to its name and version.

%ping interval and timeout value

% vim:foldmethod=marker:textwidth=60
