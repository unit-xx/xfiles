% vim:tw=60

\section{Introduction}
\label{sec:intro}


%As industry is moving towards \textit{cloud computing}, many
%players in the industry are building large and complex data
%center applications on top of clusters of commodity PCs.
%Ensuring the correctness and performance of the applications
%are critical to providing sustained service to clients.
%However, bugs have been challenging service developers and
%maintainers continually.

Distributed systems are the key enabler of most
today's Internet Services. They run over many machines,
exploit and schedule the machine resources to provide 
high-throughput, low latency services, and have to cope 
with many kinds of failures. Because of the system scale
and the critical service requirement, the design and 
implementation of these systems is prone to have bugs
that lead to unexpected execution behaviors. 
The root causes of bugs are sometimes hidden behind
the convoluted application logic, and takes long time
to find out. 

Understanding runtime behaviour of distributed systems 
is key to design verification, detection of correctness
and performance problems, and tracking down the root cause
of bugs. However, it is non-trivial, even for the developers
who implement the system, to fully understand the system 
logic. This is because of the inherent complexity 
in system design and runtime behaviors, and that 
the system is typically developed by a group of people
and keep evolving in a daily manner.
Especially, distributed systems usually adopt a layered
and tiered design, in which a user request is accomplished
through multiple stages, each stage involving a hierarchical
functional abstractions. A high-level function or user-level
task is achieved by low-level ones, which may execute
concurently and communicate across the boundaries of machines, 
processes, and threads. All of these issues contribute to the
difficulty in understanding system behaviors.

Existing approaches have shown that modeling and representing
system behaviors with hierarchical task models help developers
better understand and verify system behaviors. For example,
Pip~\cite{} allows developers to define nested task flows
with associated properties as ``expectations'', which are used 
to detect violations.
By this means, developers are able to specify and verify 
system properties at an appropriate layer. This approach is 
proven to be effective in detecting bugs, especially 
performance ones. Our previous work Scalpel~\cite{}
further avoids the manual effort of specifying task structure
by automatically inferring task models from low level
system traces. The preliminary results show that it is 
possible to infer reasonable hierarchical structures that reflect
the semantics of task flows from monitoring OS-level synchronization.
Such inferred models have helped developers find out
certain performance bugs more quickly.

Our ultimate goal is to build a tool for developers, 
testers, and system adminitrators to explore and better 
understand system behaviors through hierarchical task representation, 
\emph{without or with minimal annotation effort}.
Based on our previous work, in this paper
we further explore the idea of task model inference, 
while focusing on leveraging application-level logs instead of 
just using low level system call traces. 
The rationale is that, compared with low level traces, 
application-level logs can better reveal semantics of 
task flow and the purpose of execution pieces. 
Log entries are inserted for debugging usage 
by developers who implement the system, therefore they 
usually contain critical application-level state and events,
e.g., the begin and end of a user-level task, the critical 
steps in processing request, ect.
Leveraging this information can help our tool infer  
user-level semantics, which may hard to precisely obtained
from system call traces.

% annotation
Inferring hierarchical task models from logs needs to 
solve mainly two problems. First, logs are usually unstructured
and added by developers in ad hoc ways. We need to extract
from logs the information that effectively reveal the task 
structure. Second, there is a gap between individual log events 
and the overall task flow. A high-level task may span over
several separate log entries. We need to correlate these
log entries and further mine out the hierarchical 
task relation. Our idea is to extract the task identifiers 
that identifies user-level task instances, and further
mine out a hierarchical task structure according to
the pattern these identifiers change during the execution.
For example, as typical seen in many distributed systems, 
a log string many contain phrases like 
\texttt{"request = 1000"}, which represents the current event
belongs to the user-level task named "request" with unique ID 1000. 
We use a customizable parser and several heurisitics
to extract such identifiers as key-value pairs,
and group log entries that belong to the same task instance.
A log entry can belong to multiple tasks, and from the 
pattern how their unique IDs change, we can further
infer the task hierarchy.
For example, if there are two log entries  
\texttt{"request = 1000, operation = 1"} and 
\texttt{"request = 1000, operation = 2"}, 
we infer that these events belong to the same "request" task
(with ID 1000) but different "operation" task (with ID 1 and 2,
respectively), and that the "operation" task could be 
a child task of the "request" task. 
By summarizing the patterns from all extracted pairs,
we can effectively infer the entire hierarchical task model. 

We implemented the prototype of task inference tool.
We apply it to a distributed storage system
named Cosmos, which is similar to GFS and used by production 
team for storing web pages. Our tool infers reasonable
task structure, and we further detect a performance
bug in Cosmos guided by the infered task models. 
Our experience shows that logs are useful in reflecting
high-level system semantics, and our approach can effectively
infer task models, which can help better understand system 
design and execution behaviors.

The rest of the paper is organized as follows. We describe
our design of the inference tool in Section~\ref{}. 
We present our experience of using our tool in Section~\ref{}.
We discuss XXX in Section~\ref{}. Section~\ref{} summarizes
related work, and Section~\ref{} concludes this paper.

%In the rest paper, we present a method for automatically infer
%data hierarchy recorded in system log and use the
%information to construct task models.  We also describe our
%experiences in using the inferred task models to
%understanding the system design and debug performance
%problems.


%The hierarchical structure of tasks is often consist with
%the hierarchy of data processed by tasks. For example. In
%cosmos, data are organized as streams and streams are
%consisted of extents. By inspecting cosmos logs, task that
%processing a stream is splitted into subtasks. The first
%subtask do some stream level processing (open, etc.), and
%the following tasks process extents in the stream. But the
%task boundaries are not marked in the log.


%\notes{Main idea: data hierarchy (Session$>$opId) can be
%translated into task models (Task(Session) $:=$ Task(opId+).
%Task model can be instantiated into many distances.  We
%implement the idea on system logs.}



%Understanding runtime behaviour of the complex system is key
%to verify system design, debug its correctness and
%performance problems. By tracking task pieces and their
%causal dependencies, we can construct task flow by linking
%together pieces of its execution throughout the system.  In
%our previous work, we further developed techniques to
%automatically tracking tasks and infer hierarchical task
%models. Using the task models, developers can better
%understand the structures of components and their
%dependencies, and use debugging tools to instrument the
%system and verify the behavior of tasks at appropriate
%layers.



%By mining and leveraging data hierarchy, we can use the
%information to infer the task hierarchy. A log item which
%starts to process a stream marks the begin of a task. The
%task lasts before the log item which process another stream.
%Within the task, it is splitted into subtasks at
%extent-processing boundary.

%In this paper, we present a method for automatically infer
%data hierarchy recorded in system log and use the
%information to construct task models.  We also describe our
%experiences in using the inferred task models to
%understanding the system design and debug performance
%problems.

\comment{
It faces several challenges. First, logs are not
written in well strucutured form. Second, log contains a lot
information other than data identifer, and the noises must
be reduce automatically with best effort.

}

\comment{
and build hierarchical task models to better
understand and check system runtime behaviour.

Meanwhile, debugging correctness and performance problems
for the systems are difficult.

System log is a rich source of information for understanding
system runtime behaviour.

Researchers have proposed several techniques, but system log
is not explored much.

In this paper, we try to build hierarchical task model
through logs.

Contribution/Highlights:

Goal: Understanding system behaviour by exploiting existing
system logs

magpie: more details on low level properties d3s: customized
log point pip: manual annotation

our advangtage: 

existing log more on high level semantic
}
