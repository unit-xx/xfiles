% vim:tw=60

\section{Design}
\label{sec:design}

In this section, we describe the detailed process of mining
data hierarchy using system log.

\subsection{Extract key-value pairs}
\notes{value format are limited, we can find value part
easily, and track back in the text to find its key name.}

Different system have different log format. We describe how
to extract data from cosmos log in (key, value) format. The
principle described here can be adopted to analyze 
other kinds of log.

\begin{verbatim}
d,12/30/2008 14:51:22.949,CosmosClient,
CosmosClientDebug,SrcFile="cosmosstream.cpp"
SrcFunc="CosmosStream::StartCsmAppendJob" 
SrcLine="570" Pid="3000" Tid="9144"
TS="0x01C96A4B0702ECC2" String1="CSM append
POSTED: opId: 3, streamId:
{A8593030-1B74-4D40-B76B-52F4F7240513}, name:
cosmos://srg-sna10/vol1/graphviz-2.20.2.exe,
length: 1045679, offset: unknown,
waitForSeal: 1, status: OK"
\end{verbatim}

The cosmos log can be divided into two parts and we refer
above log item as an example. The first half has fixed
format, which records log level, time, log category and a
short title. Followed by the location of the log in source
file, and runtime parameters. The second half is contained
in \texttt{String1} field. It is written by developer and 
not well structured. It is usually interleaved by 
descriptive texts and programs states in (key, value) pairs.

It is not straightforward to extract the (key, value) pairs.
The developer can record (key, value) in several formats. In
general, it is written as \texttt{key SEPARATOR1 value
SEPARATOR2}. SEPARATOR1 and SEPARATOR2 have mutiple choices.
It is not easy to determine which word is a key, otherwise
we can do the job easily. Because a key may look similar to
an irrelevant words, such as \texttt{POSTED:} and
\texttt{opId:}.

We use a novel method to extract the (key, value) pairs. It
is based on the observation that the value part has limited
formats, such as an integer, a hexadecimal or a GUID, and it
is easy to be match by regular expression. Once the value
part is located, we can track back in the text to find its
corresponding key.

Currently, we choose to extract (key, value) pairs whose
value parts are numbers but not text. Our considerations
are: a) values as text are more descriptitive and have
little use in inferring data hierarchy in later steps. b)
SEPARATOR1 and SEPARATOR2 may be both blanks. In such cases,
text words, keys and values are messed up and they cannot be
distinguished using simple rules.

\notes{key alias, key name collision. our suggestion}

\subsection{Mining hierarchical relation of keys}

Some of the keys extracted from log have hierarchy
relations. For example, to complete a \texttt{Session}, it
is splitted into subtasks which are designated by different
\texttt{opId}. There are similar relations with
\texttt{StreadID} and \texttt{ExtentID}. We can use the
data hierarchy to deduce task hierarchy.

After the previous step, we get a stream of (key, value)
pairs in their order of appearance in log file. Activities
from multiple threads may be logged into the same log file.
We split (key, value) streams by thread and mine different
sub-streams separately. The mined results are combined
together.

We use two heuristics to mine data hierarchy among keys.
They are the necessary conditions for two keys to have
hierarchical relation.
\begin{itemize}
\item If keyS $>$ keyP, then keyS comes before keyP.
\item If keyS $>$ keyP, then there're strict 1-to-many
mapping between keyS and keyP value set.
\end{itemize}
The intuition is, if keyS $>$ keyP, one will see keyS first,
and a series of keyP with different values, and then keyS,
with another value, and then a series of keyP. 

The mining algorithm takes (key, value) stream as input and
outputs (keyS, keyP, R) tuples, indicating keyS $>$ keyP
with rank R.



reducing false positives

\notes{cannot reduce all false positives. patch 1: user
specify a subset of all keys. patch 2: user manually delete
false pairs}

\subsection{Construct hierarchical task models}
\notes{In this step, we summarize the key relations into
partial orders. and use it to construct task models.}

the input: data hierarchy in pairs

join the pairs together

the choice of different hierarchy

using the data hierarchy to construct task models

notes: the result is an approximation
