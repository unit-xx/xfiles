\section{Analysis}
\label{sec:analysis}

We analyze performance of SMON in this section. In the analysis,
we assume that SMON works in sychronized way, that is each peer
performs its epidemic activities in synchronized rounds and the
activities finish almost instantaneously. Although the real SMON
system works asynchronously, the analysis can still give
insights on performance of SMON design. We will also evaluate
SMON in real conditions (see section~\ref{sec:eval}) and
validate our analysis in this section.

% performance and scalability

SMON leverages epidemic algorithm (anti-entropy) to deploy
and maintain itself and the managed applications. The
epidemic algorithm works in three ways: called push, pull
and push-pull. Although they differ in details, existing
studies~\cite{xxx} show that each one of the three methods
can eventually infect the entire population. The studies
also show that starting from a single peer, this is achieved
in expected time (the rounds number) propotional to the log
of the population size ($\log N$).

The management activities performed by SMON chooses
different epidemic implementations (xxx: wording, modle?).
xxx use xxx, yyy works by yyy and it fits yyy model
natually.  Similiarly, zzz also uses zzz model. They are
summarized in table qqq.

self-deployment: push

upgrade: push-pull

recovery: push

enable: push

disable: pull

appman: push-pull

We analyze self-deployment in detail. Take self-deployment
as an example.

others should have the similar performance. The differences
are: barbar

% overhead number

We then analyze the overhead introduced in self-deployment.
recall why overhead occur.

verify the expression is right.


% ping overhead

binomial -> poisson distribution

% recover speed?


% vim:foldmethod=marker:textwidth=60
