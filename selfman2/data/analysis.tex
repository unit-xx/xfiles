\section{Analysis}
\label{sec:analysis}

We analyze performance of SMON in this section. In the analysis,
we assume that SMON works in sychronized way, that is each peer
performs its maintenance tasks in synchronized rounds and the
tasks finish almost instantaneously. Although the real SMON
system works asynchronously, the analysis can still give
insights on performance of SMON design. We will also evaluate
SMON in real conditions (see section~\ref{sec:eval}) and
validate our analysis.

% performance and scalability

SMON leverages epidemic algorithm (anti-entropy) to deploy
and maintain itself and the managed applications. The
epidemic algorithm works in three variants: called push,
pull and push-pull. Although they differ in details, the
basic result of epidemic theory shows that each one of the
three methods can eventually infect the entire population.
The studies also show that starting from a single peer, this
is achieved in expected time (the rounds number) propotional
to the log of the population size ($\log N$).

The maintenance tasks performed by SMON chooses
different variant of epidemic algorithm. They are summarized
in table~\ref{tbl:tasks}

self-deployment: push

upgrade: push-pull

recovery: push

enable: push

disable: pull

appman: push-pull

It is expected that all the maintenance tasks should have
good scalability and performance. Take self-deployment
process as an example. Starting from a single SMON peer, all
the target machines will be deployed with a SMON peer.
According to existing studies~\cite{Eugster2004}, the
self-deployment process can be modeled as a branching
process with population $n$. After $r$ rounds, the expected
fraction of deployed machines is:

\begin{equation*}
Y_r \approx \frac{1}{1+ne^{-r}} 
\end{equation*}

Or to say, to deploy a fix fraction of machines, the
expected round number $r$ is:

\begin{equation*}
r \approx \ln(n) - \ln(\frac{1}{Y_r} - 1)
\end{equation*}

$r$ is in the logarithm of total population $n$, which
implies good scalability of self-deployment process.

% overhead number

Another performance parameter in concern is the overhead in
the self-deployment process. Since we use epidemic
algorithm, there may be duplicated deployed SMON peers in a
machine. Although there will be only one instance running,
the duplicated deployment will waste network resource and
disk storage at some extent.

We next show that the overhead is small on average. Assuming
that there are totally $n$ machines to be deployed with SMON
peers, and $m$ machines have been deployed already. For any
one of $n-m$ un-deployed machines, it may be selected by $k,
(k <= m)$ peers as the deployment target in the next round.
The distribution of $k$ is can be expressed as binomial
distribition with parameter $(m, \frac{1}{n})$. We define
the simultaneous deployment number $k$ as the deployment
overhead. It can be seen that the expectation of $k$ is
$m/n$ and never be greater than 1. As more machines are
deployed with SMON peers, the expectation of $k$ will
approaching 1 while $m$ approaching $n$. Thus we prove that
the overhead small on average.

% ping overhead

binomial -> poisson distribution

% recover speed?


% vim:foldmethod=marker:textwidth=60
